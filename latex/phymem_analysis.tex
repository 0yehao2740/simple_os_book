\section{【实现】物理内存探测}\label{ux5b9eux73b0ux7269ux7406ux5185ux5b58ux63a2ux6d4b}

物理内存探测是在bootasm.S中实现的,相关代码很短,如下所示:

\begin{lstlisting}
probe_memory:
//对0x8000处的32位单元清零,即给位于0x8000处的
//struct e820map的结构域nr_map清零
    movl $0, 0x8000     
    xorl %ebx, %ebx    
//表示设置调用INT 15h BIOS中断后,BIOS返回的映射地址描述符的起始地址
    movw $0x8004, %di 
start_probe:
    movl $0xE820, %eax // INT 15的中断调用参数
//设置地址范围描述符的大小为20字节,其大小等于struct e820map的结构域map的大小
    movl $20, %ecx  
//设置edx为534D4150h (即4个ASCII字符“SMAP”),这是一个约定
    movl $SMAP, %edx
//调用int 0x15中断,要求BIOS返回一个用地址范围描述符表示的内存段信息
    int $0x15
//如果eflags的CF位为0,则表示还有内存段需要探测
    jnc cont
//探测有问题,结束探测
    movw $12345, 0x8000
    jmp finish_probe
cont:
//设置下一个BIOS返回的映射地址描述符的起始地址
    addw $20, %di
//递增struct e820map的结构域nr_map
    incl 0x8000
//如果INT0x15返回的ebx为零,表示探测结束,否则继续探测
    cmpl $0, %ebx
    jnz start_probe
finish_probe:
\end{lstlisting}

上述代码正常执行完毕后,在0x8000地址处保存了从BIOS中获得的内存分布信息,此信息按照struct
e820map的设置来进行填充。这部分信息将在bootloader启动ucore后,由ucore的page\_init函数来根据struct
e820map的memmap(定义了起始地址为0x8000)来完成对整个机器中的物理内存的总体管理。
