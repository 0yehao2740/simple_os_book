\section{proj8:支持页换入换出}\label{proj8ux652fux6301ux9875ux6362ux5165ux6362ux51fa}

\subsection{proj8项目组成}\label{proj8ux9879ux76eeux7ec4ux6210}

编译并运行proj8的命令如下:

\begin{lstlisting}
make
make qemu
\end{lstlisting}

则可以得到如下显示界面

\begin{lstlisting}
proj8
│   ├── driver
│   │   ├── …
│   │   ├── ide.c
│   │   └── ide.h
│   ├── fs
│   │   ├── fs.h
│   │   ├── swapfs.c
│   │   └── swapfs.h
│   ├── mm
│   │   ├── ……
│   │   ├── memlayout.h
│   │   ├── pmm.c
│   │   ├── swap.c
│   │   ├── swap.h
│   │   ├── vmm.c
│   │   └── vmm.h
│   ├── sync
│   │   └── sync.h
│   └── trap
│       ├── trap.c
│       └── ……
├── libs
│   ├── hash.c
│   └── ……
├── ……
\end{lstlisting}

相对于proj7,proj8主要修改和增加的文件如下:

\begin{itemize}
\tightlist
\item
  ide.{[}ch{]}:实现了对IDE硬盘的PIO方式的扇区读写功能,用于支持把页换入和换出硬盘。
\item
  swapfs.{[}ch{]}:根据页和硬盘扇区的映射关系,实现了在IDE硬盘上的swap文件组织,并实现了把页写入swap文件和从swap文件读入页的功能。需要ide.{[}ch{]}的支持。
\item
  swap.{[}ch{]}:参考Linux2.4的页替换策略,实现了一个简化的双链表页替换策略。
\item
  memlayout.h:修改Page等关键数据结构,支持双链页替换策略。
\item
  pmm.c:修改page\_remove\_pte函数,支持双链页替换策略。
\item
  vmm.c:修改do\_pgfault函数,支持页的换入换出。
\item
  sync.h:增加lock/unlock支持,支持页的换入换出过程不会出现race
  condition现象。
\end{itemize}

\subsection{proj8编译运行}\label{proj8ux7f16ux8bd1ux8fd0ux884c}

\begin{lstlisting}
(THU.CST) os is loading ...

Special kernel symbols:
  entry  0xc010002c (phys)
  etext  0xc010dfec (phys)
  edata  0xc012faa8 (phys)
  end    0xc0132e20 (phys)
Kernel executable memory footprint: 204KB
memory managment: buddy_pmm_manager
e820map:
  memory: 0009f400, [00000000, 0009f3ff], type = 1.
  memory: 00000c00, [0009f400, 0009ffff], type = 2.
  memory: 00010000, [000f0000, 000fffff], type = 2.
  memory: 07efd000, [00100000, 07ffcfff], type = 1.
  memory: 00003000, [07ffd000, 07ffffff], type = 2.
  memory: 00040000, [fffc0000, ffffffff], type = 2.
check_alloc_page() succeeded!
check_pgdir() succeeded!
check_boot_pgdir() succeeded!
-------------------- BEGIN --------------------
PDE(0e0) c0000000-f8000000 38000000 urw
  |-- PTE(38000) c0000000-f8000000 38000000 -rw
PDE(001) fac00000-fb000000 00400000 -rw
  |-- PTE(000e0) faf00000-fafe0000 000e0000 urw
  |-- PTE(00001) fafeb000-fafec000 00001000 -rw
--------------------- END ---------------------
check_slab() succeeded!
check_vma_struct() succeeded!
page fault at 0x00000100: K/W [no page found].
check_pgfault() succeeded!
check_vmm() succeeded.
ide 0:      10000(sectors), 'QEMU HARDDISK'.
ide 1:     262144(sectors), 'QEMU HARDDISK'.
page fault at 0x00000000: K/W [no page found].
page fault at 0x00000000: K/W [no page found].
page fault at 0x00001001: K/W [no page found].
page fault at 0x00001000: K/R [no page found].
page fault at 0x00000000: K/R [no page found].
check_swap() succeeded.
++ setup timer interrupts
100 ticks
\end{lstlisting}

check\_swap函数对ucore在proj8中建立的双链页面置换策略进行了测试,验证了其正确性,下面我们将从原理和实际实现两个方面来分析proj8中实现的页面置换算法。
