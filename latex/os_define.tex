\subsection{操作系统的定义与目标}


\paragraph{操作系统的定义}

有了对硬件的进一步了解,我们就可以给操作系统下一个更准确一些的定义。操作系统是计算机系统机构中的一个系统软件,它的职能主要有两个:对下面(也就是计算机硬件),有效地组织和管理计算机系统中的硬件资源(包括处理器、内存、硬盘、显示器、键盘、鼠标等各种外设);对上面(应用程序或用户),提供简洁的服务功能接口,屏蔽硬件管理带来的差异性和复杂性,使得应用程序和用户能够灵活、方便、有效地使用计算机。为了完成这两个职能,操作系统需要起到资源管理器的作用,能在其内部实现中安全,合理地组织,分配,使用与处理计算机中的软硬件资源,使整个计算机系统能高效可靠地运行。

\paragraph{操作系统的目标}

根据前面的介绍,我们可以看出操作系统有如下一些目标:

\begin{enumerate}
	\item 建立抽象,让上层软件和用户更方便使用;
	\item 管理软硬件资源,确保计算机系统安全可靠、高性能;
	\item 其他需求:节能、易用、可移植、实时等等。
\end{enumerate}

