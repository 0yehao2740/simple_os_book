\section{基于内核线程实现全局内存页替换机制}\label{ux57faux4e8eux5185ux6838ux7ebfux7a0bux5b9eux73b0ux5168ux5c40ux5185ux5b58ux9875ux66ffux6362ux673aux5236}

\subsection{实验目标}\label{ux5b9eux9a8cux76eeux6807}

到proj11为止,还没有能够在ucore中实现一个完整的内存页替换机制。但其实在lab2的proj8中,已经为ucore实现内存页替换机制提供了大量的支持,并在相关测试函数kern/mm/swap.c::check\_swap中进行了检查。但这个检查只是说明了proj8提供了能够完成内存页替换机制的数据结构和函数支持,即已经一砖一瓦地完成了门窗、墙壁等建筑工作,还差把相关部件完整组织起来实现成一个完整的房子。proj11就是完成这最后一步,采用内核线程来实现内存页替换机制,使得用户进程在快用完内存后,可以通过内存页替换机制把不常用的页换出到硬盘swap分区中,常用的页保存在内存中,保持系统中有足够的内存给用户进程使用。

\subsection{proj11概述}\label{proj11ux6982ux8ff0}

\subsubsection{实现描述}\label{ux5b9eux73b0ux63cfux8ff0}

proj11是lab3的第六个project。它在proj10.4的基础上实现了基于内核线程的内存页替换机制,主要扩展设计了专门用于执行内存页替换的内核线程kswapd,并增加了等待队列、扩展了进程控制块的成员变量mm的等,使得在用户进程申请存不足或系统空闲内存不足的情况下,通过执行kswapd内存线程,实现内存页替换,把不常用的页放到硬盘swap分区上,给系统提供足够的空闲空间。

\subsubsection{项目组成}\label{ux9879ux76eeux7ec4ux6210}

\begin{lstlisting}
proj11
├── ……
│   ├── mm
│   │   ├── ……
│   │   ├── pmm.c
│   │   ├── swap.c
│   │   ├── swap.h
│   │   ├── vmm.c
│   │   └── vmm.h
│   ├── process
│   │   ├──……
│   │   ├── proc.c
│   │   └── proc.h
│   └── sync
│        ├── ……
│        ├── wait.c
│        └── wait.h
└── user
    ├── cowtest.c
    ├── swaptest.c
    └── ……

17 directories, 114 files
\end{lstlisting}

相对于proj10.4,proj11在内核方面主要增加了有关kswapd内核线程和相关函数以及等待队列实现,在用户程序方面,增加测试ucore的COW实现的用户程序cowtest.c和测试内存页置换实现的swaptest.c。主要修改和增加的文件如下:

\begin{itemize}
\item
  kern/mm/pmm.c:扩展了alloc\_pages函数,使得它能够在没有获得所需空闲内存页后,进一步调用tre\_free\_pages来要求ucore释放足够的空闲页,从而再次要求所需空闲页,直到要求得到满足为止。
\item
  kern/mm/swap.{[}ch{]}:更新try\_free\_pages的实现,完成让当前进程睡眠,并唤醒kswapd内核线程,让它完成对空闲页的回收。同时实现了内核线程kswapd的执行主体kswapd\_main函数,此函数完成具体的内存页置换机制。
\item
  kern/sync/wait.{[}ch{]}:实现等待队列机制,使得内存页等资源无法得到满足的进程能够处于等待状态,并在资源得到满足后让进程继续执行。
\item
  kern/mm/vmm.{[}ch{]}:扩展了mm\_struct结构,并修改相关函数,使得所有进程的成员变量mm能够链入到全局mm\_struct结构的链表proc\_mm\_list中。
\end{itemize}

\subsubsection{编译运行}\label{ux7f16ux8bd1ux8fd0ux884c}

编译并运行proj11的命令如下:

\begin{lstlisting}
make
make qemu
\end{lstlisting}

则可以得到如下显示界面

\begin{lstlisting}
(THU.CST) os is loading ...

Special kernel symbols:
……
check_vmm() succeeded.
ide 0:      10000(sectors), 'QEMU HARDDISK'.
ide 1:     262144(sectors), 'QEMU HARDDISK'.
check_swap() succeeded.
……
++ setup timer interrupts
kernel_execve: pid = 3, name = "swaptest".
buffer size = 00500000
parent init ok.
child 9 fork ok, pid = 13.
child 8 fork ok, pid = 12.
child 7 fork ok, pid = 11.
child 6 fork ok, pid = 10.
child 5 fork ok, pid = 9.
child 4 fork ok, pid = 8.
child 3 fork ok, pid = 7.
child 2 fork ok, pid = 6.
child 1 fork ok, pid = 5.
child 0 fork ok, pid = 4.
check cow ok.
round 0
round 1
round 2
round 3
round 4
child check ok.
wait ok.
check buffer ok.
swaptest pass.
all user-mode processes have quit.
init check memory pass.
kernel panic at kern/process/proc.c:430:
    initproc exit.

Welcome to the kernel debug monitor!!
Type 'help' for a list of commands.
K>
\end{lstlisting}

表面上看不出上述输出对内存页置换算法实现的具体体现。不过通过Makefile和对swaptest.c程序的分析,还是能够看出proj11的执行与其他进程的执行不同:

\begin{lstlisting}
Makefile:
……
QEMUOPTS = -m 48m -hda $(UCOREIMG) -drive file=$(SWAPIMG),media=disk,cache=writeback
swaptest.c
……
const int size = 5 * 1024 * 1024;
char *buffer;
……
main(void){
……
(buffer = malloc(size))
……
    for (i = 0; i < pids; i ++) {
        if ((pid[i] = fork()) == 0) {
……
}
\end{lstlisting}

通过Makefile,可以看到qemu只模拟出了48MB的物理内存空间,但swaptest.c创建了10个子进程,且每个子进程都会复制全局变量buffer,且会对buffer中的所有元素进行写操作。由于每个buffer的空间大小为5MB,所以10个子进程和1个父进程的buffer所占虚拟空间总和为55MB,大于实际的48MB物理内存空间。而在操作系统设计上,用户进程的用户空间是没必要都保存在内存中,这使得必须把某些页换出到硬盘swap分区才能确保所有子进程都能正常执行完毕。为此,我们还需进一步分析proj11中ucore具体的内存页置换机制的实现和执行过程。下面将从实现方面对此进行进一步阐述。
