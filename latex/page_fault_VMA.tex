\section{proj7:支持缺页异常和VMA结构}\label{proj7ux652fux6301ux7f3aux9875ux5f02ux5e38ux548cvmaux7ed3ux6784}

\subsection{proj7项目组成}\label{proj7ux9879ux76eeux7ec4ux6210}

\begin{lstlisting}
proj7
|   |-- init
|   |   `-- init.c   
|   |-- mm

|   |   |-- pmm.c
|   |   |-- pmm.h
|   |   |-- vmm.c
|   |   `-- vmm.h
|   |-- sync
|   |   `-- sync.h
|   `-- trap
|       |-- trap.c

|-- libs
|   `-- x86.h
\end{lstlisting}

相对与proj6,proj7主要修改和增加的文件如下:

\begin{itemize}
\tightlist
\item
  init.c:在kern\_init中增加调用初始化虚存管理函数vmm\_init
\item
  pmm.{[}ch{]}:增加pgdir\_alloc\_page函数,完成分配一个空闲物理页,并设置好页表项,完成正确的虚拟地址到物理地址的转换;
\item
  trap.c:完成对缺页异常的基本操作,调用vmm.c中的do\_pgfault函数完成具体的缺页处理;
\item
  x86.h:完成对控制寄存器CR1和CR2的读操作;
\item
  vmm.{[}ch{]}:新增的文件,主要是建立vma\_struct结构,用于描述不存在的虚拟内存,并完成针对此结构的相关操作函数。
\end{itemize}

\subsection{proj7编译运行}\label{proj7ux7f16ux8bd1ux8fd0ux884c}

编译并运行proj7的命令如下:

\begin{lstlisting}
make
make qemu
\end{lstlisting}

则可以得到如下显示界面

\begin{lstlisting}
chenyu@chenyu-laptop:~/oscourse/branches/testing/chyyuu/proj7$ make qemu
(THU.CST) os is loading ...

Special kernel symbols:
  entry  0xc010002c (phys)
  etext  0xc010ae5f (phys)
  edata  0xc0127aa0 (phys)
  end    0xc0128cbc (phys)
Kernel executable memory footprint: 164KB
memory managment: buddy_pmm_manager
e820map:
  memory: 0009f400, [00000000, 0009f3ff], type = 1.
  memory: 00000c00, [0009f400, 0009ffff], type = 2.
  memory: 00010000, [000f0000, 000fffff], type = 2.
  memory: 07efd000, [00100000, 07ffcfff], type = 1.
  memory: 00003000, [07ffd000, 07ffffff], type = 2.
  memory: 00040000, [fffc0000, ffffffff], type = 2.
check_alloc_page() succeeded!
check_pgdir() succeeded!
check_boot_pgdir() succeeded!
-------------------- BEGIN --------------------
PDE(0e0) c0000000-f8000000 38000000 urw
  |-- PTE(38000) c0000000-f8000000 38000000 -rw
PDE(001) fac00000-fb000000 00400000 -rw
  |-- PTE(000e0) faf00000-fafe0000 000e0000 urw
  |-- PTE(00001) fafeb000-fafec000 00001000 -rw
--------------------- END ---------------------
check_slab() succeeded!
size of struct mm_struct is 24, size of struct vma_struct is 40
check_vma_struct() succeeded!
page fault at 0x00000100: K/W [no page found].
check_pgfault() succeeded!
check_vmm() succeeded.
++ setup timer interrupts
100 ticks
100 ticks
\end{lstlisting}

通过上图,我们可以看到ucore在check\_vma\_struct函数中完成基于vma\_struct结构的数据创建等操作,确保能够正确建立vma\_struct结构,并在成功测试后打印``check\_vma\_struct()
succeeded!'';接下来ucore创建一个描述了虚拟地址0\textasciitilde{}4K的vma\_struct结构,这个0虚拟地址起始的虚拟页没有对应的物理页,所以在实际访问这个虚拟地址的时候会产生缺页异常,中断处理例程会经过如下调用:

\begin{lstlisting}
vectorXXX(vectors.S)-->\__alltraps(trapentry.S)--> trap(trap.c)-->trap_dispatch(trap.c)—
-->pgfault_handler(trap.c)-->print_pgfault(trap.c)
\end{lstlisting}

来显示出错的位置和原因``page fault at 0x00000100: K/W {[}no page
found{]}.'',即在内核态对虚存地址0x100处执行写操作出现了缺页异常。并进一步调用do\_pgfault函数来检测时候此虚拟地址属于某个vma\_struct描述的范畴,如果是,则会分配一个物理页来对应此虚拟地址所在的虚拟页,并返回继续执行引起缺页异常的指令。如果测试能够正确执行对应的写操作指令,表明能正确处理缺页异常,则显示

\begin{lstlisting}
“check_pgfault() succeeded!”和“check_vmm() succeeded.”。
\end{lstlisting}

