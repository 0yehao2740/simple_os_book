% \iffalse meta-comment
%
% Copyright (C) 2017-2018, Haixing Hu.
% School of Physics, Nanjing University.
%
% This file may be distributed and/or modified under the conditions of the
% LaTeX Project Public License, either version 1.2 of this license or (at your
% option) any later version. The latest version of this license is in:
%
% http://www.latex-project.org/lppl.txt
%
% and version 1.2 or later is part of all distributions of LaTeX version
% 1999/12/01 or later.
%
% Home Page of the Project: http://haixing-hu.github.io/xelatex-zh-book/
%
% \fi
%
% \iffalse
%<*driver>
\ProvidesFile{zhbook.dtx}
%</driver>
%<cls>\NeedsTeXFormat{LaTeX2e}[1995/12/01]
%<cls>\ProvidesClass{zhbook}
%<cfg>\ProvidesFile{zhbook.cfg}
%<*cls>
 [2018/04/08 v1.1.3 Document Class for the Typesetting Chinese Books]
%</cls>
%<*driver>
\documentclass[12pt,a4paper,oneside]{ltxdoc}
\usepackage{dtx-style}
\EnableCrossrefs
\CodelineIndex
\GetFileInfo{zhbook.dtx}
\begin{document}
  \DocInput{zhbook.dtx}
\end{document}
%</driver>
% \fi
%
% \CheckSum{0}
% \CharacterTable
%  {Upper-case    \A\B\C\D\E\F\G\H\I\J\K\L\M\N\O\P\Q\R\S\T\U\V\W\X\Y\Z
%   Lower-case    \a\b\c\d\e\f\g\h\i\j\k\l\m\n\o\p\q\r\s\t\u\v\w\x\y\z
%   Digits        \0\1\2\3\4\5\6\7\8\9
%   Exclamation   \!     Double quote  \"     Hash (number) \#
%   Dollar        \$     Percent       \%     Ampersand     \&
%   Acute accent  \'     Left paren    \(     Right paren   \)
%   Asterisk      \*     Plus          \+     Comma         \,
%   Minus         \-     Point         \.     Solidus       \/
%   Colon         \:     Semicolon     \;     Less than     \<
%   Equals        \=     Greater than  \>     Question mark \?
%   Commercial at \@     Left bracket  \[     Backslash     \\
%   Right bracket \]     Circumflex    \^     Underscore    \_
%   Grave accent  \`     Left brace    \{     Vertical bar  \|
%   Right brace   \}     Tilde         \~}
%
% \DoNotIndex{\begin,\end,\begingroup,\endgroup}
% \DoNotIndex{\ifx,\ifdim,\ifnum,\ifcase,\else,\or,\fi}
% \DoNotIndex{\let,\def,\xdef,\newcommand,\renewcommand}
% \DoNotIndex{\expandafter,\csname,\endcsname,\relax,\protect}
% \DoNotIndex{\Huge,\huge,\LARGE,\Large,\large,\normalsize}
% \DoNotIndex{\small,\footnotesize,\scriptsize,\tiny}
% \DoNotIndex{\normalfont,\bfseries,\slshape,\interlinepenalty}
% \DoNotIndex{\hfil,\par,\vskip,\vspace,\quad}
% \DoNotIndex{\centering,\raggedright}
% \DoNotIndex{\c@secnumdepth,\@startsection,\@setfontsize}
% \DoNotIndex{\@plus,\@minus,\p@,\z@,\@m,\@M,\@ne,\m@ne,\@@par,\@dottedtocline}
% \DoNotIndex{\ ,\,,\.,\\}
% \DoNotIndex{\|}
% \DoNotIndex{\@dottedtocline}
% \DoNotIndex{\@afterindenttrue,\@arabic,\@biblabel,\@clubpenalty}
% \DoNotIndex{\@empty,\@highpenalty,\@ifnextchar,\@latex@warning,\@listI,\@listi}
% \DoNotIndex{\@mainmatterfalse,\@mainmattertrue,\@mkboth,\@nobreakfalse}
% \DoNotIndex{\@nobreaktrue,\@noitemerr,\@openbib@code,\@pnumwidth,\@restonecolfalse}
% \DoNotIndex{\@restonecoltrue,\@starttoc,\@tempcnta,\@tempdima,\@tocrmarg}
% \DoNotIndex{\@afterindenttrue,\@arabic,\@biblabel,\@clubpenalty,\@dottedtocline}
% \DoNotIndex{\@empty,\@highpenalty,\@ifnextchar,\@latex@warning,\@listI,\@listi}
% \DoNotIndex{\@mainmatterfalse,\@mainmattertrue,\@mkboth,\@nobreakfalse,\@nobreaktrue}
% \DoNotIndex{\@noitemerr,\@openbib@code,\@pnumwidth,\@restonecolfalse}
% \DoNotIndex{\@restonecoltrue,\@starttoc,\@tempcnta,\@tempdima,\@tocrmarg}
% \DoNotIndex{\abovedisplayshortskip,\abovedisplayskip,\addpenalty,\addvspace}
% \DoNotIndex{\advance,\alph,\arabic,\arraybackslash,\arraystretch,\AtBeginDocument}
% \DoNotIndex{\belowdisplayshortskip,\belowdisplayskip,\bf,\blacksquare,\bottomfraction}
% \DoNotIndex{\bullet,\c@enumiv,\c@page,\c@tocdepth,\captionsetup,\cdot,\CJKfamily}
% \DoNotIndex{\CJKglue,\CJKnumber,\CJKunderline,\CJKunderlinecolor,\ClassError}
% \DoNotIndex{\clearpage,\CurrentOption,\dagger,\day,\ddagger}
% \DoNotIndex{\DeclareGraphicsExtensions,\DeclareMathSizes,\DeclareOption}
% \DoNotIndex{\DeclareRobustCommand,\defaultfontfeatures,\DefineFNsymbolsTM}
% \DoNotIndex{\Diamondblack,\endlist,\ensuremath,\efill,\equal,\everypar}
% \DoNotIndex{\fontsize,\global,\hb@xt@,\hbox,\hfill,\hline,\hskip,\hspace,\hss}
% \DoNotIndex{\if@mainmatter,\if@restonecol,\if@twocolumn,\if@twoside,\ifodd}
% \DoNotIndex{\ifthenelse,\ignorespaces,\includegraphics,\input,\it,\item}
% \DoNotIndex{\itemsep,\kern,\l@chapter,\l@part,\labelsep,\labelwidth,\leaders}
% \DoNotIndex{\leavevmode,\leftmargin,\leftmargini,\leftmark,\leftskip,\list}
% \DoNotIndex{\LoadClass,\m@th,\makebox,\MakeUppercase,\markboth,\markright}
% \DoNotIndex{\mathparagraph,\mathsection,\mkern,\month,\multicolumn}
% \DoNotIndex{\newcolumntype,\newenvironment,\newif,\newline,\newlist,\newpage}
% \DoNotIndex{\newtheorem,\nobreak,\normalcolor,\null,\number,\onecolumn,\p@enumiv}
% \DoNotIndex{\pagestyle,\parbox,\parfillskip,\parindent,\parsep,\PassOptionsToClass}
% \DoNotIndex{\pdfbookmark,\penalty,\ProcessOptions,\punctstyle,\raisebox,\renewenvironment}
% \DoNotIndex{\RequirePackage,\RequireXeTeX,\restylefloat,\rightmargin,\rightmark,\rightskip}
% \DoNotIndex{\roman,\rule,\selectfont,,\setcounter,\setfnsymbol}
% \DoNotIndex{\setlength,\setlist,\settowidth,\sfcode,\sloppy,\square,\stretch,\tabcolsep}
% \DoNotIndex{\textasteriskcentered,\textbardbl,\textbf,\textdagger,\textdaggerdbl}
% \DoNotIndex{\textfraction,\textnormal,\textparagraph,\textsection,\textwidth}
% \DoNotIndex{\theenumiv,\theoremseparator,\theoremstyle,\theoremsymbol,\thispagestyle}
% \DoNotIndex{\titleformat,\titlespacing,\topsep,\twocolumn,\ULthickness}
% \DoNotIndex{\usecounter,\widowpenalty,\year,\color,\clubpenalty,\chaptermark}
% \DoNotIndex{\chaptertitlename,\geometry,\l@chapter,\l@part}
%
% \MakeShortVerb{\|}
% \newcommand*{\zhbook}{\texttt{Zh-Book}}
% \newcommand*{\texlive}{{\TeX}\ Live\ 2017}
% \renewcommand*{\fileversion}{1.1.3}
% \renewcommand*{\filedate}{\today}
%
% \pagestyle{empty}
% \title{\zhbook:中文书籍排版\\
%        {\XeLaTeX}模板}
% \author{{胡海星}\\
%         {\texttt{haixing.hu@qq.com}}\\
%         {南京大学物理学院}}
% \date{\fileversion\ (\filedate)}
% \maketitle
% \thispagestyle{empty}
%
% \begin{abstract}
%
% \noindent\hspace{2em}文档类{\zhbook}提供了一个中文科技书籍排版的{\XeLaTeX}
% 模板。该文档类严格遵守中文科技文献排版的相关规范和标准,底层通过
% |xeCJK|宏包支持中文。
%
% 本文档是{\zhbook}的说明文档,其中包含模板文件的设置说明以及其源代码的完全注释。
% \end{abstract}
% \clearpage
% \newpage
% \section*{{\hfill}修订历史{\hfill}}
% \begin{center}
% \noindent
% \begin{longtable}[C]{C{1.1cm}
%                      C{2cm}
%                      C{1.5cm}
%                      p{\textwidth-6.4cm}}
% \toprule
%   \textbf{版本}
%   & \textbf{日期}
%   & \textbf{修订者}
%   & \textbf{修订内容} \\
% \midrule
%  v1.0.0 & 2017/12/24 & 胡海星 & 完成第一个可工作版本 \\
%  v1.1.0 & 2018/04/06 & 胡海星 & 修改了一些常量名称;增加了对|cleveref|的中文化支持 \\
%  v1.1.1 & 2018/04/06 & 胡海星 & 修正一些小bug \\
%  v1.1.2 & 2018/04/07 & 胡海星 & 删除了对|\ref|命令的修改,所有交叉引用可直接用|\cref|命令 \\
%  v1.1.3 & 2018/04/08 & 胡海星 & 修正了一些|\crefformat|的定义\\
% \bottomrule
% \end{longtable}
% \end{center}
% \clearpage
%
% \tableofcontents
% \clearpage
%
% \pagestyle{fancy}
% \section{简介}
%
% 文档类{\zhbook}是为了帮助老师同学们撰写中文科技类教科书的模板。
% 该模板提供了一个中文书籍的{\XeLaTeX}文档类,用于生成符合中文书籍排版规范的科技类教科书。
% 该宏包的底层通过|xeCJK|宏包支持中文。
%
% 本文档将尽量完整的介绍{\zhbook}的使用方法,如有不清楚之处可以参考示例文档或
% 者与作者联系。由于作者水平有限,虽然现在的这个版本基本上满足了中文书籍的撰写需
% 求,但难免还存在不足之处,欢迎大家积极反馈意见。
%
% 本模板的编写过程中参考了以下代码和文档,这里一并向这些代码和文档的作者表示感谢:
%
% \begin{itemize}
% \item 胡海星. \textsl{南京大学学位书籍{\LaTeX}模板}. \url{http://haixing-hu.github.io/nju-thesis/}.
% \item 杨文博. \textsl{南京大学学位书籍{\LaTeX}模板}. \url{https://code.google.com/p/zhbook/}.
% \item 薛瑞尼. \textsl{清华大学学位书籍{\LaTeX}模板}.
% \item 胡卫谊. \textsl{武汉理工大学学位书籍{\LaTeX}模板}.
% \item 吴凯. \textsl{GBT7714-2005NLang.bst}. v1.0 beta 2. 2006.
% \item \textsl{CTeX宏包}. \url{http://www.ctex.org}.
% \item The {\LaTeX}3 Project. \textsl{{\LaTeXe} for class and package writers}.
% \item Frank Mittelbach, Michel Gooseens. \textsl{The {\LaTeX} Companion}. 2nd ed.
% \item Scott Pakin. \textsl{How to Package Your {\LaTeX} Package}.
% \url{http://www.iitg.ernet.in/trivedi/LatexHelp/Latex%20Manual/dtxtut.pdf}.
% \item Oren Patashnik. \textsl{Designing \BibTeX Styles}. 1988.
% \end{itemize}
%
% \section{遵循的要求和标准}
%
% {\zhbook}所遵循的中华人民共和国国家标准如下:
% \begin{itemize}
% \item \href{https://github.com/Haixing-Hu/typesetting-standard/raw/master/%E5%9B%BE%E4%B9%A6%E3%80%81%E6%9C%9F%E5%88%8A%E3%80%81%E8%AE%BA%E6%96%87%E7%9A%84%E7%BC%96%E6%8E%92/%E3%80%90GB:T%207714-2005%E3%80%91%E6%96%87%E5%90%8E%E5%8F%82%E8%80%83%E6%96%87%E7%8C%AE%E8%91%97%E5%BD%95%E8%A7%84%E5%88%99.pdf}%
% {\std{GB/T 7714-2005}\textsl{文后参考文献著录规则}}
% \item \href{https://github.com/Haixing-Hu/typesetting-standard/raw/master/%E5%9B%BE%E4%B9%A6%E3%80%81%E6%9C%9F%E5%88%8A%E3%80%81%E8%AE%BA%E6%96%87%E7%9A%84%E7%BC%96%E6%8E%92/%E3%80%90GB:T%207713.1-2006%E3%80%91%E5%AD%A6%E4%BD%8D%E8%AE%BA%E6%96%87%E7%BC%96%E5%86%99%E8%A7%84%E5%88%99.pdf}%
% {\std{GB/T 7713.1-2006}\textsl{学位书籍编写规则}}%
% \item \href{https://github.com/Haixing-Hu/typesetting-standard/raw/master/%E5%9B%BE%E4%B9%A6%E3%80%81%E6%9C%9F%E5%88%8A%E3%80%81%E8%AE%BA%E6%96%87%E7%9A%84%E7%BC%96%E6%8E%92/%E3%80%90GB:T%207713.3-2009%E3%80%91%E7%A7%91%E6%8A%80%E6%8A%A5%E5%91%8A%E7%BC%96%E5%86%99%E8%A7%84%E5%88%99.pdf}%
% {\std{GB/T 7713.3-2009}\textsl{科技报告编写规则}}
% \item \href{https://github.com/Haixing-Hu/typesetting-standard/raw/master/%E5%9B%BE%E4%B9%A6%E3%80%81%E6%9C%9F%E5%88%8A%E3%80%81%E8%AE%BA%E6%96%87%E7%9A%84%E7%BC%96%E6%8E%92/%E3%80%90GB:T%207713-1987%E3%80%91%E7%A7%91%E5%AD%A6%E6%8A%80%E6%9C%AF%E6%8A%A5%E5%91%8A%E3%80%81%E5%AD%A6%E4%BD%8D%E8%AE%BA%E6%96%87%E5%92%8C%E5%AD%A6%E6%9C%AF%E8%AE%BA%E6%96%87%E7%9A%84%E7%BC%96%E5%86%99%E6%A0%BC%E5%BC%8F.pdf}%
% {\std{GB/T 7713-1987}\textsl{科学技术报告、学位书籍和学术书籍的编写格式}},
% 该标准已被%
% \href{https://github.com/Haixing-Hu/typesetting-standard/raw/master/%E5%9B%BE%E4%B9%A6%E3%80%81%E6%9C%9F%E5%88%8A%E3%80%81%E8%AE%BA%E6%96%87%E7%9A%84%E7%BC%96%E6%8E%92/%E3%80%90GB:T%207713.1-2006%E3%80%91%E5%AD%A6%E4%BD%8D%E8%AE%BA%E6%96%87%E7%BC%96%E5%86%99%E8%A7%84%E5%88%99.pdf}%
% {\std{GB/T 7713.1-2006}}和%
% \href{https://github.com/Haixing-Hu/typesetting-standard/raw/master/%E5%9B%BE%E4%B9%A6%E3%80%81%E6%9C%9F%E5%88%8A%E3%80%81%E8%AE%BA%E6%96%87%E7%9A%84%E7%BC%96%E6%8E%92/%E3%80%90GB:T%207713.3-2009%E3%80%91%E7%A7%91%E6%8A%80%E6%8A%A5%E5%91%8A%E7%BC%96%E5%86%99%E8%A7%84%E5%88%99.pdf}%
% {\std{GB/T 7713.3-2009}}部分替代
% \item \href{https://github.com/Haixing-Hu/typesetting-standard/raw/master/%E5%9B%BE%E4%B9%A6%E3%80%81%E6%9C%9F%E5%88%8A%E3%80%81%E8%AE%BA%E6%96%87%E7%9A%84%E7%BC%96%E6%8E%92/%E3%80%90GB:T%207156-2003%E3%80%91%E6%96%87%E7%8C%AE%E4%BF%9D%E5%AF%86%E7%AD%89%E7%BA%A7%E4%BB%A3%E7%A0%81%E4%B8%8E%E6%A0%87%E8%AF%86.pdf}%
% {\std{GB/T 7156-2003}\textsl{文献保密等级代码与标识}}
% \item \href{https://github.com/Haixing-Hu/typesetting-standard/raw/master/%E6%95%B0%E5%AD%97%E6%96%87%E5%AD%97/%E3%80%90GB:T%2016159-2012%E3%80%91%E6%B1%89%E8%AF%AD%E6%8B%BC%E9%9F%B3%E6%AD%A3%E8%AF%8D%E6%B3%95%E5%9F%BA%E6%9C%AC%E8%A7%84%E5%88%99.pdf}%
% {\std{GB/T 16159-2012}\textsl{汉语拼音正词法基本规则}},
% 该标准取代了%
% \href{https://github.com/Haixing-Hu/typesetting-standard/raw/master/%E6%95%B0%E5%AD%97%E6%96%87%E5%AD%97/%E3%80%90GB:T%2016159-1996%E3%80%91%E6%B1%89%E8%AF%AD%E6%8B%BC%E9%9F%B3%E6%AD%A3%E8%AF%8D%E6%B3%95%E5%9F%BA%E6%9C%AC%E8%A7%84%E5%88%99.pdf}%
% {\std{GB/T 16159-1996}}
% \item \href{https://github.com/Haixing-Hu/typesetting-standard/raw/master/%E5%9B%BE%E4%B9%A6%E3%80%81%E6%9C%9F%E5%88%8A%E3%80%81%E8%AE%BA%E6%96%87%E7%9A%84%E7%BC%96%E6%8E%92/%E3%80%90CY:T%2035-2001%E3%80%91%E7%A7%91%E6%8A%80%E6%96%87%E7%8C%AE%E7%9A%84%E7%AB%A0%E8%8A%82%E7%BC%96%E5%8F%B7%E6%96%B9%E6%B3%95.pdf}%
% {\std{CY/T 35-2001}\textsl{科技文献的章节编号方法}}
% \end{itemize}
%
% 上述要求和标准的电子版均可在对应的链接地址中找到。
%
% \section{安装}
%
% \subsection{下载}
%
% 可在{\zhbook}项目主页上下载最新版本,亦可在代码库主页上反馈bug和意见建议:
% \begin{itemize}
% \item 项目主页:\url{http://haixing-hu.github.io/xelatex-zh-book/}
% \item 代码库主页:\url{http://haixing-hu.github.io/xelatex-zh-book/}
% \end{itemize}
%
% \subsection{模板的组成部分}
%
% \begin{table}
%   \centering\noindent
%   \begin{tabular*}{\textwidth}{p{4cm}p{\textwidth-4.5cm}}
%     \toprule
%     \textbf{文件(夹)}        & \textbf{功能描述}\\
%     \midrule
%     |zhbook.ins|             & 模板驱动文件 \\
%     |zhbook.dtx|             & 模板文档代码的混合文件\\
%     |zhbook.cls|             & 模板类文件\\
%     |zhbook.cfg|             & 模板配置文件\\
%     |gbt7714-2005.bst|          & 符合国标\std{GB/T 7714-2005}的参考文献样式文件\\
%     |dtx-style.sty|             & 用户手册样式文件\\
%     \hline
%     |sample.tex|                & 示例文档,亦可作为书籍的基本模板 \\
%     |sample.bib|                & 示例文档的参考文献数据库 \\
%     |figures/|                  & 示例文档图片目录\\
%     \hline
%     |Makefile|                  & make 脚本 \\
%     |get_texmf_dir.sh|          & 获取本地|textmf|目录路径的脚本\\
%     \hline
%     |README.md|                 & 说明文件 \\
%     |zhbook.pdf|                & 用户手册(本文档)\\
%     \bottomrule
%   \end{tabular*}
%   \caption{{\zhbook}的主要文件及其功能}\label{table:component}
% \end{table}
%
% 表\ref{table:component}列出了{\zhbook}的主要文件及其功能。其中|zhbook.cls|,
% |zhbook.cfg|和|dtx-sty.sty|可以由|zhbook.ins|和|zhbook.dtx|生成,但为
% 了降低新手用户的使用难度,故将其一起发布。
%
% \subsection{准备工作}
%
% 表\ref{table:dependence}列出了{\zhbook}模板用到的宏包。这些包在常见的{\TeX}系
% 统中都有(推荐使用{\texlive}),如果没有请到\url{www.ctan.org}下载。
%
% \begin{table}
%   \centering\noindent
%   \begin{tabular*}{\textwidth}{@{\extracolsep{\fill}}*{6}{l}}
%   \hline
%     |ifxetex|  & |indentfirst| & |xeCJK|    & |lastpage|  & |geometry|  & |graphicx| \\
%     |subfig|   & |caption|     & |float|    & |array|     & |longtable| & |booktabs| \\
%     |multirow| & |hyperref|    & |enumitem| & |xcolor|    & |amsmath|   & |amsfonts| \\
%     |amsthm|   & |amssymb|     & |bm|       & |mathrsfs|  & |txfonts|   & |pifont|  \\
%     |setspace| & |wasysym|     & |hypernat| & |fancyhdr|  & |natbib|    & |tabularx| \\
%     |titlesec| & |glossaries|  & |ifthen|   & |makeidx|   & |footmisc|  &  |CJKnumb| \\
%     |url|      & |etoolbox|    & |commath|  & |mathtools| & |blindtext| & |diagbox| \\
%     |cleveref| &               &            &             &             &  \\
%   \hline
%   \end{tabular*}
%   \caption{{\zhbook}用到的宏包}\label{table:dependence}
% \end{table}
%
% \subsection{推荐的{\TeX}系统}
%
% 本模板当前版本v{\fileversion}{\ }({\filedate})在{\texlive}下编写,尚未在其他
% {\TeX}系统上测试。因此推荐用户使用{\texlive}。其安装包可以在下述网址下载:
% \begin{center}
% \url{http://tug.org/texlive/}
% \end{center}
%
% \begin{note}
% 由于本模板采用{\XeLaTeX}引擎处理,所以{\TeX}源文件应使用\textbf{UTF-8}编码。
% \end{note}
%
% \subsection{开始安装}
%
% \subsubsection{生成模板}
%
% 默认的发行包中已经包含了所有文件,可以直接使用。如果对如何由|*.dtx|生成模板文件以及模板文
% 档不感兴趣,请跳过本小节。
%
% 模板解压缩后生成文件夹|zhbook-VERSION|,其中|VERSION|为版本号。该文件夹中包括:
% \begin{itemize}
% \item 模板源文件:|zhbook.ins|和|zhbook.dtx|
% \item 参考文献样式:|gbt7714-2005.bst|
% \item 示例文档:|sample.tex|、|sample.bib|和|figure|目录
% \end{itemize}
%
% 在使用之前需要先生成模板文件和配置文件,具体命令细节请参考|README|和|Makefile|。下面是
% 在Linux或Mac系统中生成模板所需执行的|shell|命令:
%
% \begin{shell}
% $ cd zhbook-VERSION
% # 清理以前执行make生成的旧文件
% $ make clean
% # 生成 zhbook.cls 和 zhbook.cfg
% $ make cls
% # 生成文档类手册
% $ make doc
% # 生成样例文档
% $ make sample
% \end{shell}
%
% \subsubsection{安装到{\TeX}系统中}
%
% 假设当前{\TeX}系统的texmf-local目录为|${TEXMFLOCAL}|。下面是在Linux或Mac系统中将模
% 板安装到本机的{\TeX}系统中所需执行的|shell|命令:
%
% \begin{shell}
% $ cd zhbook-VERSION
% # 建立zhbook文档类目录
% $ mkdir -p ${TEXMFLOCAL}/tex/latex/zhbook
% # 复制zhbook文档类文件
% $ cp zhbook.cls ${TEXMFLOCAL}/tex/latex/zhbook/
% $ cp zhbook.cfg ${TEXMFLOCAL}/tex/latex/zhbook/
% # 复制zhbook文档类的源码,此过程可选
% $ cp zhbook.ins ${TEXMFLOCAL}/tex/latex/zhbook/
% $ cp zhbook.dtx ${TEXMFLOCAL}/tex/latex/zhbook/
% # 创建本地的BibTeX样式文件目录
% $ mkdir -p ${TEXMFLOCAL}/bibtex/bst
% # 复制GB/T 7714-2005参考文献样式
% $ cp gbt7714-2005.bst ${TEXMFLOCAL}/bibtex/bst/
% # 建立zhbook文档类手册目录
% $ mkdir -p ${TETEXMFLOCALXMF}/doc/latex/zhbook
% # 复制zhbook文档类手册和示例文档
% $ cp zhbook.pdf ${TEXMFLOCAL}/doc/latex/zhbook/
% $ cp sample.pdf ${TEXMFLOCAL}/doc/latex/zhbook/
% # 刷新tex文件名数据库
% $ texhash
% \end{shell}
%
% \begin{note}
% 上面的某些命令可能需要管理员权限,或通过|sudo|执行。
% \end{note}
%
% 当然,也可以直接使用|Makefile|提供的|install|操作进行安装:
% \begin{shell}
% $ sudo make install
% \end{shell}
%
% \begin{note}
% |Makefile|使用了脚本|get_texmf_dir.sh|来获取当前机器上所安装的{\TeX}系统的本地
% |textmf|目录(通常是{\TeX}安装目录下的|textmf-local|目录)。用户最好在执行
% |make install|之前先执行一下|get_texmf_dir.sh|脚本,看看输出的目录路径是否正确。
% 如不正确,可以手工修改|Makefile|中对|TEXMFLOCAL|变量的定义。
% \end{note}
%
% \section{使用说明}
%
% 本手册假定用户已经能处理一般的{\LaTeX}文档,并对{\BibTeX}有一定了解。如果你从来没有接
% 触过{\TeX}和{\LaTeX},建议先学习相关的基础知识。
%
% \subsection{\zhbook{} 示例文件}
%
% 模板核心文件只有三个:|zhbook.cls|,|zhbook.cfg|和 |gbt7714-2005.bst|,但
% 是如果没有示例文档用户会发现很难下手。所以推荐新用户从模板自带的示例文档入手,
% 里面包括了文档写作用到的所有命令及其使用方法,只需要用自己的内容进行相应替换就
% 可以。对于不清楚的命令可以查阅本手册。具体内容可以参考模板附带的|sample.tex|和
% |sample.bib|。
%
% \subsection{选项}
%
% 本文档类提供了一些选项以方便使用:
% \begin{description}
% \item[winfonts, linuxfonts, macfonts, adobefonts] |winfonts|选项使得文档使
%   用Windows系统提供的字体;|linuxfonts|选项使得文档使用Linux系统提供的字
%   体;|macfonts|选项使得文档使用Mac系统提供的字体;|adobefonts|选项使得文档使
%   用Adobe提供的OTF中文字体,一般来说OTF字体的显示效果要优于ttf字体。
%   默认选项是|adobefonts|。
% \begin{example}
% \documentclass[winfonts]{zhbook}
% \end{example}
%   表\ref{table:fontnames}中列出了默认配置下使用不同字体选项时所采用的实际字体
%   名称。系统中必须正确安装了相应的字体才能正常编译文档。\\
%   Adobe的宋体和黑体可以在其公司网站免费下载:
%   \begin{center}
%   \url{http://www.adobe.com/support/downloads/detail.jsp?ftpID=4421}
%   \end{center}
%   楷体无免费下载,但在网上可以找到。下面的网址提供了一个打包下载的地址:
%   \begin{center}
%   \url{http://tinker-bot.googlecode.com/files/cfonts.tar.gz}
%   \end{center}
%   \begin{table}
%     \centering\noindent
%     \begin{tabular}[t]{ccccc}
%     \toprule
%           & \textbf{adobefonts} &  \textbf{winfonts} & \textbf{linuxfonts} & \textbf{macfonts} \\
%     \midrule
%     \textbf{宋体} & {Adobe Song Std}  & {SimSun} & {AR PL SungtiL GB} &  {STSong} \\
%     \textbf{黑体} & {Adobe Heiti Std} & {SimHei} & {WenQuanYi Zen Hei Mono} &  {STHeiti} \\
%     \textbf{楷书} & {Adobe Kaiti Std} & {KaiTi}  & {AR PL KaitiM GB} & {STKaiti} \\
%     \textbf{仿宋体} & {Adobe Fangsong Std} & {FangSong} & {STFangSong} & {STFangSong} \\
%     \bottomrule
%     \end{tabular}
%     \caption{默认配置下不同字体选项所使用的实际字体名称}
%     \label{table:fontnames}
%   \end{table}
%
% \end{description}
%
% 本文档类不再提供对字号、字体和单双面打印的选择选项。因为国内各出版社对中文科技类教科书的排版基本上都要求
% 使用小四号宋体,双面打印。
%
% \subsection{命令和环境}
%
% 文档类中的命令和环境分为三类:一是格式控制,二是内容替换,三是文档结构。格式控制如字体、字
% 号、字距和行距等;内容替换如文档名称、作者、项目、编号等;文档结构如中文摘要、中文关键词、
% 英文摘要、英文关键词、作者简历、致谢等。
%
% \subsubsection{格式控制命令}
%
% \myentry{中文字体}
% \DescribeMacro{\songti}
% \DescribeMacro{\heiti}
% \DescribeMacro{\kaishu}
% \DescribeMacro{\fangsong}
% 可采用下述命令选择中文字体
% \begin{itemize}
% \item \cs{songti} 切换宋体
% \item \cs{heiti} 切换黑体
% \item \cs{kaishu} 切换楷书
% \item \cs{fangsong} 切换仿宋体
% \end{itemize}
%
% \begin{example}
% {\songti 乾:元,亨,利贞}
% {\heiti 九二,见龙在田,利见大人}
% {\kaishu 九三,君子终日乾乾,夕惕若,厉,无咎}
% {\fangsong 九四,或跃在渊,无咎}
% \end{example}
%
% \myentry{字号}
% \DescribeMacro{\zihao}
% \cs{zihao}命令可用于选择字号。其语法为:
% \begin{syntax}
% \cs{zihao}\marg{n}
% \end{syntax}
% 其中参数\meta{n}为要使用的字号;在\meta{n}前加负号$-$表示小号字体。目前提供的字号包括:
% \begin{itemize}
% \item 初号(|\zihao{0}|)、小初号(|\zihao{-0}|)
% \item 一号(|\zihao{1}|)、小一号(|\zihao{-1}|)
% \item 二号(|\zihao{2}|)、小二号(|\zihao{-2}|)
% \item 三号(|\zihao{3}|)、小三号(|\zihao{-3}|)
% \item 四号(|\zihao{4}|)、小四号(|\zihao{-4}|)
% \item 五号(|\zihao{5}|)、小五号(|\zihao{-5}|)
% \item 六号(|\zihao{6}|)、小六号(|\zihao{-6}|)
% \item 七号(|\zihao{7}|)
% \item 八号(|\zihao{8}|)
% \end{itemize}
%
% \begin{example}
% {\zihao{2} 二号} {\zihao{3} 三号} {\zihao{4} 四号} {\zihao{-4} 小四}
% \end{example}
%
% \myentry{字距}
% \DescribeMacro{\ziju}
% \cs{ziju}命令可用于更改汉字之间默认的距离。其语法为:
% \begin{syntax}
% \cs{ziju}\marg{width}
% \end{syntax}
% 其中的参数\meta{width}只要是合格的{\TeX}距离即可。
%
% \begin{example}
% {\ziju{4bp}调整字距示例}
% \end{example}
%
% \myentry{缩进}
% \DescribeMacro{\indent}
% \DescribeMacro{\noindent}
% \cs{indent}命令将当前行正常的缩进两个汉字字宽的距离,同时在汉字大小和字距改变的情况都
% 可以自动修改缩进距离。
%
% \cs{noindent}则取消缩进。
%
% \myentry{破折号}
% \DescribeMacro{\zhdash}
% 中文破折号在CJK-{\LaTeX}里没有很好的处理,我们平时输入的都是两个小短线,比如这样,
% ``{中国——中华人民共和国}''。这不符合中文习惯。所以这里定义了一个命令生成更好看的破折号。
% 不过这似乎不是一个好的解决办法,比如不能用在\cs{section}等命令中使用。简单的办法是可以
% 提供一个不带破折号的段标题:
% \begin{syntax}
% \cs{section}\oarg{没有破折号精简标题}\marg{带破折号的标题}
% \end{syntax}
%
% \begin{example}
% 测试--中文破折号
% 测试{\zhdash}中文破折号
% \end{example}
%
% 上述例子的显示效果分别如下:
% \begin{itemize}
% \item 测试--中文破折号
% \item 测试{\zhdash}中文破折号
% \end{itemize}
%
% \subsubsection{中文封面内容替换命令}
%
% 本节描述书籍中文封面的内容替换命令。
%
% \myentry{书籍标题}
% \DescribeMacro{\title}
% 命令\cs{title}用于设置书籍的中文标题。
%
% \begin{example}
% \title{量子行走理论、实现和应用}
% \end{example}
%
% \begin{note}
% \cs{title}的参数中不可换行,也不能使用\cs{thanks}脚注。
% \end{note}
%
% \myentry{书籍标题分行}
% \DescribeMacro{\titlea}
% \DescribeMacro{\titleb}
% 命令\cs{titlea}和\cs{titleb}用于在书籍标题很长时,设置分行的书籍中文标题。
% 其中\cs{titlea}设置书籍标题的第一行,\cs{titleb}设置书籍标题的第二行。
%
% \begin{example}
% \titlea{半轻衰变$D^+\to \omega(\phi)e^+\nu_e$的研究}
% \titleb{和弱衰变$J/\psi \to D_s^{(*)-}e^+\nu_e$的寻找}
% \end{example}
%
% \begin{note}
% \cs{title}的参数中不可换行,也不能使用\cs{thanks}脚注。
% \end{note}
%
% \myentry{作者姓名}
% \DescribeMacro{\author}
% 命令\cs{author}用于设置书籍作者的姓名。此属性必须被设置。
%
% \begin{example}
% \author{张三}
% \author{张三、李四、王五等}
% \end{example}
%
% \begin{note}
% \cs{author}的参数中不可换行,也不能使用\cs{thanks}脚注。
% \end{note}
%
% \myentry{出版社名称}
% \DescribeMacro{\publisher}
% 命令\cs{publisher}用于设置书籍出版社的名称。此命令为可选,默认值为``南京大学出版社''。
%
% \begin{example}
% \publisher{南京大学出版社}
% \end{example}
%
% \begin{note}
% \cs{publisher}的参数中不可换行。
% \end{note}
%
% \myentry{印刷日期}
% \DescribeMacro{\date}
% 命令\cs{date}用于设置书籍的定稿日期,该日期将出现在中文封面下方以及书脊下方。可设
% 置年、月、日。此属性可选,默认值为最后一次编译时的日期,精确到日。
%
% \begin{example}
% \date{2013年5月27日}
% \end{example}
%
% \subsubsection{英文封面内容替换命令}
%
% 本节描述书籍的英文封面的内容替换命令。
%
% \myentry{书籍标题}
% \DescribeMacro{\englishtitle}
% 命令\cs{englishtitle}用于设置书籍的英文标题。此属性必须被设置。
%
% \begin{example}
% \englishtitle{Network Models of Data Centers based on the Small World Theory}
% \end{example}
%
% \begin{note}
% \cs{englishtitle}的参数中不可换行,也不能使用\cs{thanks}脚注。
% \end{note}
%
% \subsubsection{中文摘要页内容替换命令}
%
% 本节描述书籍的中文摘要页的内容替换命令。
%
% \myentry{标题及副标题}
% \DescribeMacro{\abstracttitlea}
% \DescribeMacro{\abstracttitleb}
% 命令\cs{abstracttitlea}和\cs{abstracttitleb}分别用于设置中文摘要页面的书籍标题
% 及副标题的第一行和第二行。\cs{abstracttitlea}命令为可选,其默认值为使用\cs{title}
% 命令所设置的书籍标题;\cs{abstracttitleb}命令为可选,其默认值为空白。这两个命令
% 是为了让用户在书籍标题较长时手动进行分割换行。
%
% \begin{example}
% \abstracttitlea{基于小世界理论的}
% \abstracttitleb{数据中心网络模型研究}
% \end{example}
%
% \begin{note}
% \cs{abstracttitlea}和\cs{abstracttitleb}命令的参数中都不能出现换行。
% \end{note}
%
% \subsubsection{英文摘要页内容替换命令}
%
% 本节描述书籍的英文摘要页的内容替换命令。
%
% \myentry{标题及副标题}
% \DescribeMacro{\englishabstracttitlea}
% \DescribeMacro{\englishabstracttitleb}
% 命令\cs{abstracttitlea}和\cs{abstracttitleb}分别用于设置英文摘要页面的书籍标题
% 及副标题的第一行和第二行。\cs{englishabstracttitlea}命令为可选,其默认值为使用
% \cs{englishtitle}命令所设置的书籍英文标题;\cs{englishabstracttitleb}命令为可
% 选,其默认值为空白。这两个命令是为了让用户在书籍标题较长时手动进行分割换行。
%
% \begin{example}
% \englishabstracttitlea{A Network Model of Data Centers}
% \englishabstracttitleb{Based on the Small World Theory}
% \end{example}
%
% \begin{note}
% \cs{englishabstracttitlea}和\cs{englishabstracttitleb}命令的参数中都不能换行。
% \end{note}
%
% \subsubsection{文档结构命令和环境}
%
% 本节描述书籍中可能用到的其他文档结构命令和环境。
%
% \myentry{生成中文封面}
% \DescribeMacro{\maketitle}
% 命令\cs{maketitle}用于生成书籍的中文封面。此命令必须被用在{\TeX}文档的
% \cs{begin{document}}命令之后和\cs{frontmatter}命令之前。
%
% \begin{example}
% \maketitle
% \end{example}
%
% \myentry{生成英文封面}
% \DescribeMacro{\makeenglishtitle}
% 命令\cs{makeenglishtitle}用于生成书籍的英文封面。此命令必须被用在{\TeX}文档的
% \cs{begin{document}}命令之后和\cs{frontmatter}命令之前。
%
% \begin{example}
% \makeenglishtitle
% \end{example}
%
% \myentry{内容简介}
% \DescribeEnv{abstract}
% \env{abstract}为内容简介环境。此环境必须被用在{\TeX}文档的\cs{frontmatter}命令之后和
% \cs{mainmatter}命令之前。
%
% \begin{example}
% \begin{abstract}
% 本文基于小世界理论,研究了数据中心的网络模型。………………
% \end{abstract}
% \end{example}
%
% \myentry{中文关键词}
% \DescribeMacro{\keywords}
% 命令\cs{keywords}用于设置中文关键词。此命令必须被用在\env{abstract}环境中。关键词
% 之间用中文全角分号隔开。
%
% \begin{example}
% \begin{abstract}
% 本文基于小世界理论,研究了数据中心的网络模型。………………
% \keywords{数据中心;网络模型;小世界理论}
% \end{abstract}
% \end{example}
%
% \myentry{序言}
% \DescribeEnv{prologue}
% \env{prologue}为书籍序言环境。此环境必须被用在{\TeX}文档的
% \env{abstract}环境之后和\cs{tableofcontents}命令之前。
%
% \myentry{前言}
% \DescribeEnv{preface}
% \env{preface}为书籍前言环境。此环境必须被用在{\TeX}文档的
% \env{abstract}环境之后和\cs{tableofcontents}命令之前。
%
% \begin{example}
% \begin{preface}
%  复杂网络的研究可上溯到20世纪60年代对ER网络的研究。90年后代随着Internet
%  的发展,以及对人类社会、通信网络、生物网络、社交网络等各领域研究的深入,
%  发现了小世界网络和无尺度现象等普适现象与方法。对复杂网络的定性定量的科
%  学理解和分析,已成为如今网络时代科学研究的一个重点课题。
%
%  在此背景下,由于云计算时代的到来,本文针对面向云计算的数据中心网络基础
%  设施设计中的若干问题,进行了几方面的研究。本文的创造性研究成果主要如下
%  几方面:
%
%  ………
%
%
%  \vspace{1cm}
%  \begin{flushright}
%   韦小宝\\
%   2013年夏于南京大学南苑
%  \end{flushright}
% \end{preface}
% \end{example}
%
% \myentry{目次}
% \DescribeMacro{\tableofcontents}
% 命令\cs{tableofcontents}用于生成书籍目次。此命令必须被用在{\TeX}文档的
% \env{preface}环境之后和\cs{mainmatter}命令之前。
%
% \begin{example}
% \tableofcontents
% \end{example}
%
% \myentry{附表清单}
% \DescribeMacro{\listoftables}
% 命令\cs{listoftables}用于生成书籍的附表清单。此命令为可选命令。此命令必须被用在
% {\TeX}文档的\cs{tableofcontents}命令之后和\cs{mainmatter}命令之前。
%
% \begin{example}
% \listoftables
% \end{example}
%
% \myentry{插图清单}
% \DescribeMacro{\listoffigures}
% 命令\cs{listoffigures}用于生成书籍插图清单。此命令为可选命令。此命令必须被用在
% {\TeX}文档的\cs{tableofcontents}命令之后和\cs{mainmatter}命令之前。
%
% \begin{example}
% \listoffigures
% \end{example}
%
% \myentry{致谢章节}
% \DescribeEnv{acknowledgement}
% \env{acknowledgement}环境用于生成致谢章节。此环境必须被用在书籍的最后一章(通
% 常是“结论”章节)之后以及{\TeX}文档的\cs{appendix}命令和\cs{backmatter}命令之前。
%
% \begin{example}
% \begin{acknowledgement}
% 首先感谢我的母亲韦春花对我的支持。其次感谢我的导师陈近南对我的精心指导和热心帮助。接
% 下来,感谢我的师兄茅十八和风际中,他们阅读了我的书籍草稿并提出了很有价值的修改建议。
%
% 最后,感谢我亲爱的老婆们:双儿、苏荃、阿珂、沐剑屏、曾柔、建宁公主、方怡,感谢你们在
% 生活上对我无微不至的关怀和照顾。
% \end{acknowledgement}
% \end{example}
%
% \subsubsection{其它命令和环境}
%
% \myentry{列表环境}
% \DescribeEnv{itemize}
% \DescribeEnv{enumerate}
% \DescribeEnv{description}
% 为了适合中文习惯,{\zhbook}文档类使用|paralist|宏包重新定义了|itemize|、
% |enumerate|和|description|这三个常用的列表环境。一方面满足了多余空间的清楚,另
% 一方面可以自己指定标签的样式和符号。
%
% 使用的细节请参看|paralist|文档,此处不再赘述。
%
% \subsection{数学环境}
%
% {\zhbook}宏包预定义了一些数学定理环境,如表\ref{table:math-env}所示。
%
% \begin{table}
% \noindent\centering
% \begin{tabular}{*{7}{l}}
%   \hline
%   axiom     & theorem   & definition & proposition & lemma      & conjecture & notation \\
%   公理       & 定理      & 定义        & 命题         & 引理       & 猜想        & 记号 \\
%   \hline
%   proof     & corollary & example    & exercise    & assumption & remark     & problem \\
%   证明       & 推论      & 例子        & 练习         & 假设        & 评注       & 问题\\
%   \hline
%   postulate &  hypothesis  &  principle  &  algorithm &           &      &  \\
%   公设       &  假说        &   定律       &  算法       &         &        & \\
%   \hline
% \end{tabular}
% \caption{预定义的数学定理环境}\label{table:math-env}
% \end{table}
%
% 例如:
% \begin{example}
% \begin{definition}
% 小世界网络是指其平均路径长度和其节点总数成对数关系的网络。
% \end{definition}
% \end{example}
% 上述代码将产生(自动编号):
% \begin{flushleft}
% {\heiti 定义~1.1~~~} {小世界网络是指其平均路径长度和其节点总数成对数关系的网络。}
% \end{flushleft}
%
% 列举出来的数学环境毕竟是有限的,如果想用{\heiti 胡说}这样的数学环境,那么很容易定义:
% \begin{example}
% \newtheorem{nonsense}{胡说}[chapter]
% \end{example}
%
% 然后这样使用:
% \begin{example}
% \begin{nonsense}
% 契丹武士要来中原夺武林秘笈。\zhdash 慕容博
% \end{nonsense}
% \end{example}
% 上述代码将产生(自动编号):
% \begin{flushleft}
% {\heiti 胡说~1.1~~~} {契丹武士要来中原夺武林秘笈。\zhdash 慕容博}
% \end{flushleft}
%
% \subsection{自定义以及其它}
%
% 文档类的配置文件|zhbook.cfg|中定义了很多固定词汇,一般无须修改。如果有特殊需求,
% 推荐在导言区使用\cs{renewcommand}。当然,导言区里可以直接使用中文。
%
% \section{实现细节}
%
% \subsection{定义选项}
%
% {\zhbook}宏包的默认选项为|adobefonts|。
%    \begin{macrocode}
%<*cls>
\newif\ifzhbook@adobefonts\zhbook@adobefontstrue
\newif\ifzhbook@winfonts\zhbook@winfontsfalse
\newif\ifzhbook@linuxfonts\zhbook@linuxfontsfalse
\newif\ifzhbook@macfonts\zhbook@macfontsfalse
\newif\ifzhbook@backinfo\zhbook@backinfotrue
\newif\ifzhbook@phd\zhbook@phdfalse
\newif\ifzhbook@master\zhbook@masterfalse
\newif\ifzhbook@bachelor\zhbook@bachelorfalse
\DeclareOption{adobefonts}{\zhbook@adobefontstrue
  \zhbook@winfontsfalse
  \zhbook@linuxfontsfalse
  \zhbook@macfontsfalse}
\DeclareOption{winfonts}{\zhbook@winfontstrue
  \zhbook@adobefontsfalse
  \zhbook@linuxfontsfalse
  \zhbook@macfontsfalse}
\DeclareOption{linuxfonts}{\zhbook@linuxfontstrue
  \zhbook@adobefontsfalse
  \zhbook@winfontsfalse
  \zhbook@macfontsfalse}
\DeclareOption{macfonts}{\zhbook@macfontstrue
  \zhbook@adobefontsfalse
  \zhbook@winfontsfalse
  \zhbook@linuxfontsfalse}
\DeclareOption{nobackinfo}{\zhbook@backinfofalse}
\DeclareOption{phd}{\zhbook@phdtrue
  \zhbook@masterfalse
  \zhbook@bachelorfalse}
\DeclareOption{master}{\zhbook@mastertrue
  \zhbook@phdfalse
  \zhbook@bachelorfalse}
\DeclareOption{bachelor}{\zhbook@bachelortrue
  \zhbook@phdfalse
  \zhbook@masterfalse}
%    \end{macrocode}
%
% 把没有定义的选项传递给底层的文档类,在这里为|book|。
%
%    \begin{macrocode}
\DeclareOption*{\PassOptionsToClass{\CurrentOption}{book}}
%    \end{macrocode}
%
% 处理选项:
%    \begin{macrocode}
\ProcessOptions\relax
%    \end{macrocode}
%
% \subsection{底层文档类}
%
% 文档基于{\LaTeX}的标准|book|类。正文使用小四字号(对应于12.05pt,这里近似使用12pt),
% 纸张使用A4,双面打印。
%    \begin{macrocode}
\LoadClass[12pt,a4paper,twoside]{book}
%    \end{macrocode}
%
% \subsection{装载宏包}
%
% 使用本文档类所写的文档需要使用{\XeLaTeX}引擎处理,因此首先要检查引擎是否正确。
%    \begin{macrocode}
\RequirePackage{ifxetex}
\RequireXeTeX
%    \end{macrocode}
%
% 使用|lastpage|宏包来获得最后一页的页码,从而生成“第3页,共20页”这样的页码标签。
%    \begin{macrocode}
\RequirePackage{lastpage}
%    \end{macrocode}
%
% 使用|geometry|宏包定义页面布局,定义段间距。
%    \begin{macrocode}
\RequirePackage{geometry}
%    \end{macrocode}
%
% 使用|titlesec|宏包设置标题格式。
%    \begin{macrocode}
\RequirePackage{titlesec}
%    \end{macrocode}
%
% 使用|graphicx|宏包支持插入图片。
%    \begin{macrocode}
\RequirePackage{graphicx}
%    \end{macrocode}
%
% 如果插入的图片没有指定扩展名,那么依次搜索下面的扩展名所对应的文件
%    \begin{macrocode}
\DeclareGraphicsExtensions{.pdf,.eps,.jpg,.png}
%    \end{macrocode}
%
% |caption2|宏包已经不再推荐使用,改用新的|caption|宏包处理浮动图形和表格的标题
% 样式。
%    \begin{macrocode}
\RequirePackage{caption}
%    \end{macrocode}
%
% |float|宏包为浮动图形和表格环境提供了一个H选项,强制将其放在当前位置。
%    \begin{macrocode}
\RequirePackage{float}
%    \end{macrocode}
%
% |subfigure|宏包已经不再推荐使用,改用新的|subfig|宏包支持插入子图和子表。
%    \begin{macrocode}
\RequirePackage{subfig}
%    \end{macrocode}
%
% 使用|array|宏包扩展表格的列选项。
%    \begin{macrocode}
\RequirePackage{array}
%    \end{macrocode}
%
% 使用|longtable|宏包处理长表格。
%    \begin{macrocode}
\RequirePackage{longtable}
%    \end{macrocode}
%
% |booktabs|宏包可生成三线表,支持\cs{toprule},\cs{midrule},\cs{bottomrulle}等命令。
%    \begin{macrocode}
\RequirePackage{booktabs}
%    \end{macrocode}
%
% |multirow|宏包支持在表格中跨行。
%    \begin{macrocode}
\RequirePackage{multirow}
%    \end{macrocode}
%
% |enumitem|宏包支持自定义列表环境。
%    \begin{macrocode}
\RequirePackage{enumitem}
%    \end{macrocode}
%
% |xcolor|宏包提供色彩控制。
%    \begin{macrocode}
\RequirePackage{xcolor}
%    \end{macrocode}
%
% |commath|和|mathtools|宏包提供常用数学公式支持。
%    \begin{macrocode}
\RequirePackage{commath}
\RequirePackage{mathtools}
%    \end{macrocode}
%
% |amsmath|宏包提供数学公式支持。
%    \begin{macrocode}
\RequirePackage{amsmath}
%    \end{macrocode}
%
% |amsthm|宏包支持自定义数学定理环境。
%    \begin{macrocode}
\RequirePackage{amsthm}
%    \end{macrocode}
%
% |amsfonts|宏包、|amssymb|宏包、|bm|宏包和|mathrsfs|宏包提供数学符号和字体支持。
%    \begin{macrocode}
\RequirePackage{amsfonts}
\RequirePackage{amssymb}
\RequirePackage{bm}
\RequirePackage{mathrsfs}
%    \end{macrocode}
%
% |wasysym|宏包提供特殊符号支持。
%    \begin{macrocode}
\RequirePackage{wasysym}
%    \end{macrocode}
%
% |pifont|宏包提供带圈的数字符号。
%    \begin{macrocode}
\RequirePackage{pifont}
%    \end{macrocode}
%
% |txfonts|宏包用自己的typewriter字体替换系统Courier字体,它必须在{\AmSTeX}之后引入。
%    \begin{macrocode}
\RequirePackage{txfonts}
%    \end{macrocode}
%
% |setspace|宏包支持行距控制。它需要在|hyperref|宏包之前加载,避免脚注超链接失效。
%    \begin{macrocode}
\RequirePackage{setspace}
%    \end{macrocode}
%
% |fancyhdr|宏包支持自定义页眉页脚。
%    \begin{macrocode}
\RequirePackage{fancyhdr}
%    \end{macrocode}
%
% |shortvrb|提供了一个\cs{MakeShortVerb}命令,可将某个符号定义为\cs{verb}命令的缩写。
%    \begin{macrocode}
\RequirePackage{shortvrb}
%    \end{macrocode}
%
% 使用|xltxtra|宏包来获取{\XeLaTeX}的符号。
%    \begin{macrocode}
\RequirePackage{xltxtra}
%    \end{macrocode}
%
% 使用|xeCJK|宏包处理中文。宏包选项|CJKnumber|表示调用|CJKnumber|宏包处理中文数
% 字;|CJKchecksingle|表示避免单个汉字单独占一行。|xeCJK|宏包必须放在|amssymb|之后,
% 否则会有冲突。
% \begin{note}
%   因为我们将使用黑体作为粗体,使用楷体作为斜体,因此载入|xeCJK|宏包时不需要开启
%   |BoldFont|选项和|SlantFont|;否则的话,|xeCJK|会自动生成宋体的粗体和斜体,而那会
%   非常难看。
% \end{note}
% \begin{note}
% 由于TeX Live升级到2014版后,直接用|xeCJK|的|CJKnumber|选项会出现bug,我们需单独导入
% |CJKnumb|宏包;但|xeCJK|的|CJKnumber|选项依然需要,否则在Tex Live 2012下编译会报错。
% \end{note}
%    \begin{macrocode}
\RequirePackage[CJKnumber,CJKchecksingle]{xeCJK}
\RequirePackage{CJKnumb}
%    \end{macrocode}
%
% 让{\XeLaTeX}能够处理dash的惯例(使用"--"和"---"获得较长的dash)。
%    \begin{macrocode}
\defaultfontfeatures{Mapping=tex-text}
%    \end{macrocode}
%
% 设置中文标点格式,使用|plain|方案。其他可选方案参见|xeCJK|文档。
%    \begin{macrocode}
\punctstyle{plain}
%    \end{macrocode}
%
% |xeCJKfntef|宏包提供了中文下划线命令\cs{CJKunderline},它将在制作书籍封面时用到。
%    \begin{macrocode}
\RequirePackage{xeCJKfntef}
%    \end{macrocode}
%
% 设置中文下划线颜色为黑色。
%    \begin{macrocode}
\xeCJKsetup{ underline/format = \color{black} }
%    \end{macrocode}
%
% 使用|indentfirst|宏包支持首行缩进。
%    \begin{macrocode}
%\RequirePackage{indentfirst}
%    \end{macrocode}
%
% |url|宏包支持超链接排版,我们为它提供|hyphens|选项,从而使得长链接可在连字符处自动折行。
% 注意|url|宏包必须在|hyperref|宏包之前载入,否则其选项不起作用。
%    \begin{macrocode}
\RequirePackage[hyphens]{url}
%    \end{macrocode}
%
% |hyperref|宏包可根据交叉引用生成超链接,同时生成PDF文档的书签。
%    \begin{macrocode}
\RequirePackage{hyperref}
%    \end{macrocode}
%
% 设置|hyperref|宏包参数。
%    \begin{macrocode}
\hypersetup{%
    unicode=true,
    hyperfootnotes=true,
    hyperindex=true,
    pageanchor=true,
    CJKbookmarks=true,
    bookmarksnumbered=true,
    bookmarksopen=true,
    bookmarksopenlevel=0,
    breaklinks=true,
    colorlinks=false,
    plainpages=false,
    pdfpagelabels,
    pdfborder=0 0 0%
}
%    \end{macrocode}
%
% 设置URL样式,使其与上下文一致。
%    \begin{macrocode}
\urlstyle{same}
%    \end{macrocode}
%
% 使用宏包|cleveref|,可以通过|\cref{label}|进行交叉引用,而无需手动添加引用对象的名称。
% |\cref{label}|可以自动添加引用对象的名称。例如,|参见式\eqref{eq1}|,|参见定义\ref{thm1}|
% 可以简化为|参见\cref{eq1}|,|参见\cref{thm1}|。
%    \begin{macrocode}
\RequirePackage{cleveref}
%    \end{macrocode}
%
% 美化参考文献排序和引用格式的宏包|natbib|。
%    \begin{macrocode}
\RequirePackage[sort&compress,numbers]{natbib}
%    \end{macrocode}
%
% |hypernat|可以让|hyperref|和|natbib|混合使用,但它需要放在这两者之后。
%    \begin{macrocode}
\RequirePackage{hypernat}
%    \end{macrocode}
%
% |tabularx|宏包支持自动扩展的列宽,但它需要在|hyperref|之后引入才不会导致正文
% 的footnote的超链接失效。
%    \begin{macrocode}
\RequirePackage{tabularx}
%    \end{macrocode}
%
% |makeidx|宏包支持建立索引。
%    \begin{macrocode}
\RequirePackage{makeidx}
%    \end{macrocode}
%
% |glossaries|宏包可用于制作术语表。但该宏包必须在|hyperref|之后载入。
%    \begin{macrocode}
\RequirePackage{glossaries}
%    \end{macrocode}
%
% |ifthen|宏包提供了\cs{ifthenelse}命令,本文档类将使用该命令定义一些其他命令。
%    \begin{macrocode}
\RequirePackage{ifthen}
%    \end{macrocode}
%
% |footmisc|宏包提供了对脚注样式的控制功能。
%    \begin{macrocode}
\RequirePackage[perpage,symbol*]{footmisc}
%    \end{macrocode}
%
% |etoolbox|宏包提供了一些工具宏。
%    \begin{macrocode}
\RequirePackage{etoolbox}
%    \end{macrocode}
%
% |diagbox|宏包提供斜线划分表格单元格的简单办法。
%    \begin{macrocode}
\RequirePackage{diagbox}
%    \end{macrocode}
%
% |blindtext|宏包提供了命令用于生成随机的占位文字,从而在草稿阶段可以看到排版的整体效果。
% |math|参数允许|blindtext|生成带数学公式的占位文字。
%    \begin{macrocode}
\RequirePackage[math]{blindtext}
%    \end{macrocode}
%
%</cls>
%
% \subsection{字符串常量定义}
%
% 定义书籍中各章节的中文标题名称字符串常量:
%    \begin{macrocode}
%<*cfg>
\newcommand*{\zhbook@cap@cover}{封面}
\newcommand*{\zhbook@cap@abstract}{内容简介}
\newcommand*{\zhbook@cap@contents}{目\hspace{2em}次}
\newcommand*{\zhbook@cap@revisionhistory}{修订历史}
\newcommand*{\zhbook@cap@listfigure}{插图清单}
\newcommand*{\zhbook@cap@listtable}{附表清单}
\newcommand*{\zhbook@cap@listsymbol}{符号清单}
\newcommand*{\zhbook@cap@listequation}{公式清单}
\newcommand*{\zhbook@cap@equation}{公式}
\newcommand*{\zhbook@cap@bib}{参考文献}
\newcommand*{\zhbook@cap@glossary}{术\hspace{0.5em}语\hspace{0.5em}表}
\newcommand*{\zhbook@cap@index}{索\hspace{2em}引}
\newcommand*{\zhbook@cap@figure}{图}
\newcommand*{\zhbook@cap@table}{表}
\newcommand*{\zhbook@cap@prologue}{序\hspace{2em}言}
\newcommand*{\zhbook@cap@preface}{前\hspace{2em}言}
\newcommand*{\zhbook@cap@acknowledgement}{致\hspace{2em}谢}
\newcommand*{\zhbook@cap@appendix}{附录\thechapter}
%    \end{macrocode}
%
% 定义用于重定义\cs{chaptername}命令的常量。若当前所处位置是文档的|mainmatter|部分,ze
% 将其定义为``第XX章''的形式,否则将其定义为空字符串。
%    \begin{macrocode}
\newcommand*{\zhbook@cap@chapter}{%
  \if@mainmatter{第\CJKnumber{\thechapter}章}\fi%
}
%    \end{macrocode}
%
% 定义常用数学定理环境的字符串常量:
%    \begin{macrocode}
\newcommand*{\zhbook@cap@definition}{定义}
\newcommand*{\zhbook@cap@notation}{记号}
\newcommand*{\zhbook@cap@theorem}{定理}
\newcommand*{\zhbook@cap@lemma}{引理}
\newcommand*{\zhbook@cap@corollary}{推论}
\newcommand*{\zhbook@cap@proposition}{命题}
\newcommand*{\zhbook@cap@fact}{事实}
\newcommand*{\zhbook@cap@assumption}{假设}
\newcommand*{\zhbook@cap@conjecture}{猜想}
\newcommand*{\zhbook@cap@hypothesis}{假说}
\newcommand*{\zhbook@cap@axiom}{公理}
\newcommand*{\zhbook@cap@postulate}{公设}
\newcommand*{\zhbook@cap@principle}{定律}
\newcommand*{\zhbook@cap@problem}{问题}
\newcommand*{\zhbook@cap@exercise}{习题}
\newcommand*{\zhbook@cap@example}{例}
\newcommand*{\zhbook@cap@remark}{评注}
\newcommand*{\zhbook@cap@proof}{证明}
\newcommand*{\zhbook@cap@solution}{解}
\newcommand*{\zhbook@cap@algorithm}{算法}
\newcommand*{\zhbook@cap@result}{结果}
%    \end{macrocode}
%
% 定义自定义列表环境的字符串常量:
%    \begin{macrocode}
\newcommand*{\zhbook@cap@case}{情况}
\newcommand*{\zhbook@cap@subcase}{子情况}
\newcommand*{\zhbook@cap@step}{步骤}
\newcommand*{\zhbook@cap@substep}{子步骤}
%    \end{macrocode}
%
% 定义日期中的中文字符:
%    \begin{macrocode}
\newcommand*{\zhbook@cap@year}{年}
\newcommand*{\zhbook@cap@month}{月}
\newcommand*{\zhbook@cap@day}{日}
\newcommand*{\zhbook@cap@to}{至}
%    \end{macrocode}
%
% \subsection{格式控制常量定义}
%
% 定义Windows下宋体、黑体、楷书和仿宋体四种中文字体的名称。默认采用微软字体。
%    \begin{macrocode}
\newcommand*{\zhbook@zhfn@songti@win}{SimSun}
\newcommand*{\zhbook@zhfn@heiti@win}{SimHei}
\newcommand*{\zhbook@zhfn@kaishu@win}{KaiTi}
\newcommand*{\zhbook@zhfn@fangsong@win}{FangSong}
%    \end{macrocode}
%
% 定义Windows下英文字体的名称。默认采用Windows自带的字体。
%    \begin{macrocode}
\newcommand*{\zhbook@enfn@main@win}{Times New Roman}
\newcommand*{\zhbook@enfn@sans@win}{Arial}
\newcommand*{\zhbook@enfn@mono@win}{Courier New}
%    \end{macrocode}
%
% 定义Linux下宋体、黑体、楷书和仿宋体四种中文字体的名称。默认采用文鼎宋体、楷体;
% 文泉黑体;以及华文仿宋体(需要单独安装)。
%    \begin{macrocode}
\newcommand*{\zhbook@zhfn@songti@linux}{AR PL SungtiL GB}
\newcommand*{\zhbook@zhfn@heiti@linux}{WenQuanYi Zen Hei Mono}
\newcommand*{\zhbook@zhfn@kaishu@linux}{AR PL KaitiM GB}
\newcommand*{\zhbook@zhfn@fangsong@linux}{STFangSong}
%    \end{macrocode}
%
% 定义Linux下英文字体的名称。默认采用的字体若未安装请自行安装。
%    \begin{macrocode}
%\newcommand*{\zhbook@enfn@main@linux}{Times}
%\newcommand*{\zhbook@enfn@sans@linux}{Helvetica}
%\newcommand*{\zhbook@enfn@mono@linux}{Courier}
\newcommand*{\zhbook@enfn@main@linux}{Times New Roman}
\newcommand*{\zhbook@enfn@sans@linux}{Arial}
\newcommand*{\zhbook@enfn@mono@linux}{Courier New}
%    \end{macrocode}
%
% 定义Mac下宋体、黑体、楷书和仿宋体四种中文字体的名称。默认采用华文字体。
%    \begin{macrocode}
\newcommand*{\zhbook@zhfn@songti@mac}{STSong}
\newcommand*{\zhbook@zhfn@heiti@mac}{STHeiti}
\newcommand*{\zhbook@zhfn@kaishu@mac}{STKaiti}
\newcommand*{\zhbook@zhfn@fangsong@mac}{STFangsong}
%    \end{macrocode}
%
% 定义Mac下英文字体的名称。默认采用Mac自带的字体。
%    \begin{macrocode}
\newcommand*{\zhbook@enfn@main@mac}{Times}
\newcommand*{\zhbook@enfn@sans@mac}{Helvetica}
\newcommand*{\zhbook@enfn@mono@mac}{Courier}
%    \end{macrocode}
%
% 定义Adoble提供的宋体、黑体、楷书和仿宋体四种中文字体的名称。Adoble的宋体、黑体和
% 仿宋体可以在其网站免费下载,地址为
% \begin{center}
%  \url{http://www.adobe.com/support/downloads/detail.jsp?ftpID=4421}
% \end{center}
% 但Adobe的楷体只随Adobe Creative Suite软件提供。不过,所有Adobe中文字体都可以在这里
% 打包下载:
% \begin{center}
% \url{http://tinker-bot.googlecode.com/files/cfonts.tar.gz}
% \end{center}
%    \begin{macrocode}
\newcommand*{\zhbook@zhfn@songti@adobe}{Adobe Song Std}
\newcommand*{\zhbook@zhfn@heiti@adobe}{Adobe Heiti Std}
\newcommand*{\zhbook@zhfn@kaishu@adobe}{Adobe Kaiti Std}
\newcommand*{\zhbook@zhfn@fangsong@adobe}{Adobe Fangsong Std}
%    \end{macrocode}
%
% 定义英文字体的名称。默认采用Mac自带的字体。
%    \begin{macrocode}
%\newcommand*{\zhbook@enfn@main@adobe}{Times}
%\newcommand*{\zhbook@enfn@sans@adobe}{Helvetica}
%\newcommand*{\zhbook@enfn@mono@adobe}{Courier}
\newcommand*{\zhbook@enfn@main@adobe}{Times New Roman}
\newcommand*{\zhbook@enfn@sans@adobe}{Arial}
\newcommand*{\zhbook@enfn@mono@adobe}{Courier New}
%    \end{macrocode}
%
% \subsection{中文化cleveref宏包}
%
% 中文化|cleveref|宏包所用到的字符串常量:
%    \begin{macrocode}
\crefformat{equation}{#2式~\textnormal{(}#1\textnormal{)}~#3}
\crefformat{figure}{#2图~#1~#3}
\crefformat{table}{#2表~#1~#3}
\crefformat{listing}{#2清单~#1~#3}
\crefformat{page}{#2第~#1~页#3}
\crefformat{line}{#2第~#1~行#3}
\crefformat{part}{#2第~#1~部#3}
\crefformat{chapter}{#2第~#1~章#3}
\crefformat{section}{#2~#1~节#3}
\crefformat{subsection}{#2~#1~节#3}
\crefformat{appendix}{#2附录~#1~#3}
\crefformat{enumi}{#2第~#1~点#3}
\crefformat{footnote}{#2脚注~#1~#3}
\crefformat{definition}{#2定义~#1~#3}
\crefformat{notation}{#2记号~#1~#3}
\crefformat{theorem}{#2定理~#1~#3}
\crefformat{lemma}{#2引理~#1~#3}
\crefformat{corollary}{#2推论~#1~#3}
\crefformat{proposition}{#2命题~#1~#3}
\crefformat{fact}{#2事实~#1~#3}
\crefformat{assumption}{#2假设~#1~#3}
\crefformat{conjecture}{#2猜想~#1~#3}
\crefformat{hypothesis}{#2假说~#1~#3}
\crefformat{axiom}{#2公理~#1~#3}
\crefformat{postulate}{#2公设~#1~#3}
\crefformat{principle}{#2定律~#1~#3}
\crefformat{problem}{#2问题~#1~#3}
\crefformat{exercise}{#2习题~#1~#3}
\crefformat{example}{#2例~#1~#3}
\crefformat{remark}{#2评注~#1~#3}
\crefformat{proof}{#2证明~#1~#3}
\crefformat{solution}{#2解~#1~#3}
\crefformat{algorithm}{#2算法~#1~#3}
\crefformat{result}{#2结果~#1~#3}
%    \end{macrocode}
%</cfg>
%
% \subsection{载入字符串常量配置}
%
% 在进行其他配置之前先载入预定义的字符串常量配置。
%    \begin{macrocode}
%<*cls>
%%
%% Copyright (C) 2017-2018, Haixing Hu.
%% School of Physics, Nanjing University.
%%
%% This file may be distributed and/or modified under the conditions of the
%% LaTeX Project Public License, either version 1.2 of this license or (at your
%% option) any later version. The latest version of this license is in:
%%
%% http://www.latex-project.org/lppl.txt
%%
%% and version 1.2 or later is part of all distributions of LaTeX version
%% 1999/12/01 or later.
%%
%% Home Page of the Project: http://haixing-hu.github.io/xelatex-zh-book/
%%
%% ---------------- Start of the docstrip commands --------------------

\input docstrip

\askforoverwritefalse
%% \askonceonly
%% \showprogress
\keepsilent

\usedir{tex/latex/zhbook}

\preamble
This is a generated file.

Copyright (C) 2017-2018, Haixing Hu.
School of Physics, Nanjing University.

Home Page of the Project: http://haixing-hu.github.io/xelatex-zh-book/

It may be distributed and/or modified under the conditions of the LaTeX Project
Public License, either version 1.2 of this license or (at your option) any
later version.  The latest version of this license is in

   http://www.latex-project.org/lppl.txt

and version 1.2 or later is part of all distributions of LaTeX version
1999/12/01 or later.

To produce the documentation run the original source files ending with `.dtx'
through LaTeX.
\endpreamble

\declarepreamble\cfgpreamble
This is a generated file.

Copyright (C) 2017-2018, Haixing Hu.
School of Physics, Nanjing University.

Home Page of the Project: http://haixing-hu.github.io/xelatex-zh-book/

It may be distributed and/or modified under the conditions of the LaTeX Project
Public License, either version 1.2 of this license or (at your option) any
later version.  The latest version of this license is in

   http://www.latex-project.org/lppl.txt

and version 1.2 or later is part of all distributions of LaTeX version
1999/12/01 or later.

This is the configuration file of the zhbook package with XeLaTeX.
\endpreamble

\declarepreamble\dtxstypreamble
This is a generated file.

Copyright (C) 2017-2018, Haixing Hu.
School of Physics, Nanjing University.

Home Page of the Project: http://haixing-hu.github.io/xelatex-zh-book/

It may be distributed and/or modified under the conditions of the LaTeX Project
Public License, either version 1.2 of this license or (at your option) any
later version.  The latest version of this license is in

   http://www.latex-project.org/lppl.txt

and version 1.2 or later is part of all distributions of LaTeX version
1999/12/01 or later.

This is the manual-style file of the zhbook package with XeLaTeX.
\endpreamble

\generate{\file{\jobname.cls}{\from{\jobname.dtx}{cls}}%
          \usepreamble\cfgpreamble%
          \file{\jobname.cfg}{\from{\jobname.dtx}{cfg}}%
          \usepreamble\dtxstypreamble%
          \file{dtx-style.sty}{\from{\jobname.dtx}{dtx-style}}%
}

\ifToplevel{
\Msg{***********************************************************}
\Msg{*}
\Msg{* To finish the installation you have to move the following}
\Msg{* files into a directory searched by TeX:}
\Msg{*}
\Msg{* The recommended directory is TEXMF/tex/latex/zhbook}
\Msg{*}
\Msg{* \space\space zhbook.cls}
\Msg{* \space\space zhbook.cfg}
\Msg{*}
\Msg{* To produce the documentation run the file}
\Msg{* `zhbook.dtx' through LaTeX.}
\Msg{*}
\Msg{* Happy TeXing}
\Msg{***********************************************************}
}

\endbatchfile

%    \end{macrocode}
%
% \subsection{字体设置}
%
% 首先根据文档选项选择正确的中文字体名称。
%    \begin{macrocode}
\ifzhbook@adobefonts
  \newcommand*{\zhbook@zhfn@songti}{\zhbook@zhfn@songti@adobe}
  \newcommand*{\zhbook@zhfn@heiti}{\zhbook@zhfn@heiti@adobe}
  \newcommand*{\zhbook@zhfn@kaishu}{\zhbook@zhfn@kaishu@adobe}
  \newcommand*{\zhbook@zhfn@fangsong}{\zhbook@zhfn@fangsong@adobe}
  \newcommand*{\zhbook@enfn@main}{\zhbook@enfn@main@adobe}
  \newcommand*{\zhbook@enfn@sans}{\zhbook@enfn@sans@adobe}
  \newcommand*{\zhbook@enfn@mono}{\zhbook@enfn@mono@adobe}
\else
  \ifzhbook@winfonts
      \newcommand*{\zhbook@zhfn@songti}{\zhbook@zhfn@songti@win}
      \newcommand*{\zhbook@zhfn@heiti}{\zhbook@zhfn@heiti@win}
      \newcommand*{\zhbook@zhfn@kaishu}{\zhbook@zhfn@kaishu@win}
      \newcommand*{\zhbook@zhfn@fangsong}{\zhbook@zhfn@fangsong@win}
      \newcommand*{\zhbook@enfn@main}{\zhbook@enfn@main@win}
      \newcommand*{\zhbook@enfn@sans}{\zhbook@enfn@sans@win}
      \newcommand*{\zhbook@enfn@mono}{\zhbook@enfn@mono@win}
  \else
    \ifzhbook@linuxfonts
      \newcommand*{\zhbook@zhfn@songti}{\zhbook@zhfn@songti@linux}
      \newcommand*{\zhbook@zhfn@heiti}{\zhbook@zhfn@heiti@linux}
      \newcommand*{\zhbook@zhfn@kaishu}{\zhbook@zhfn@kaishu@linux}
      \newcommand*{\zhbook@zhfn@fangsong}{\zhbook@zhfn@fangsong@linux}
      \newcommand*{\zhbook@enfn@main}{\zhbook@enfn@main@linux}
      \newcommand*{\zhbook@enfn@sans}{\zhbook@enfn@sans@linux}
      \newcommand*{\zhbook@enfn@mono}{\zhbook@enfn@mono@linux}
    \else
       \ifzhbook@macfonts
          \newcommand*{\zhbook@zhfn@songti}{\zhbook@zhfn@songti@mac}
          \newcommand*{\zhbook@zhfn@heiti}{\zhbook@zhfn@heiti@mac}
          \newcommand*{\zhbook@zhfn@kaishu}{\zhbook@zhfn@kaishu@mac}
          \newcommand*{\zhbook@zhfn@fangsong}{\zhbook@zhfn@fangsong@mac}
          \newcommand*{\zhbook@enfn@main}{\zhbook@enfn@main@mac}
          \newcommand*{\zhbook@enfn@sans}{\zhbook@enfn@sans@mac}
          \newcommand*{\zhbook@enfn@mono}{\zhbook@enfn@mono@mac}
       \else
         \ClassError{zhbook}{No fonts was selected.}{}
       \fi
    \fi
  \fi
\fi
%    \end{macrocode}
%
% 接下来定义文档使用的中文字体:
%    \begin{macrocode}
\setCJKfamilyfont{song}{\zhbook@zhfn@songti}
\setCJKfamilyfont{hei}{\zhbook@zhfn@heiti}
\setCJKfamilyfont{kai}{\zhbook@zhfn@kaishu}
\setCJKfamilyfont{fangsong}{\zhbook@zhfn@fangsong}
\setCJKmainfont[BoldFont={\zhbook@zhfn@heiti},%
                ItalicFont={\zhbook@zhfn@kaishu}]%
               {\zhbook@zhfn@songti}
\setCJKsansfont{\zhbook@zhfn@heiti}
\setCJKmonofont{\zhbook@zhfn@fangsong}
%    \end{macrocode}
%
% 定义文档使用的英文字体。
%    \begin{macrocode}
\setmainfont{\zhbook@enfn@main}
\setsansfont{\zhbook@enfn@sans}
\setmonofont{\zhbook@enfn@mono}
%    \end{macrocode}
%
% 定义中文字体选择命令。
%    \begin{macrocode}
\newcommand*{\songti}{\CJKfamily{song}}
\newcommand*{\heiti}{\CJKfamily{hei}}
\newcommand*{\kaishu}{\CJKfamily{kai}}
\newcommand*{\fangsong}{\CJKfamily{fangsong}}
%    \end{macrocode}
%
% \begin{table}
%   \centering
%   \subtable[科学出版社编写的《著译编辑手册》(1994年)中定义的中文字号大小]{
%     \label{table:fontsize:standard}
%     \noindent
%     \begin{tabular}{ccc}
%       \toprule
%       \textbf{字号}  &   \textbf{大小(pt)} & \textbf{大小(mm)}   \\
%       \midrule
%       七号  &    5.25  &    1.845 \\
%       六号  &    7.875 &    2.768 \\
%       小五  &    9     &    3.163 \\
%       五号  &    10.5  &    3.69  \\
%       小四  &    12    &    4.2175 \\
%       四号  &    13.75 &    4.83   \\
%       三号  &    15.75 &    5.53  \\
%       二号  &    21    &    7.38  \\
%       一号  &    27.5  &    9.48  \\
%       小初  &    36    &    12.65 \\
%       初号  &    42    &    14.76 \\
%       \bottomrule
%     \end{tabular}
%   }
%   \qquad
%   \subtable[Microsoft Word中定义的中文字号大小,其中$1$bp=$72.27/72$pt]{
%     \label{table:fontsize:word}
%     \noindent
%     \begin{tabular}{cccc}
%       \toprule
%       \textbf{字号}  & \textbf{大小(bp)} & \textbf{大小(mm)} & \textbf{大小(pt)}   \\
%       \midrule
%       初号  & 42     & 14.82 & 42.1575  \\
%       小初  & 36     & 12.70 & 36.135   \\
%       一号  & 26     & 9.17  & 26.0975  \\
%       小一  & 24     & 8.47  & 24.09    \\
%       二号  & 22     & 7.76  & 22.0825  \\
%       小二  & 18     & 6.35  & 18.0675  \\
%       三号  & 16     & 5.64  & 16.06    \\
%       小三  & 15     & 5.29  & 15.05625 \\
%       四号  & 14     & 4.94  & 14.0525  \\
%       小四  & 12     & 4.23  & 12.045   \\
%       五号  & 10.5   & 3.70  & 10.59375 \\
%       小五  & 9      & 3.18  & 9.03375  \\
%       六号  & 7.5    & 2.56  &            \\
%       小六  & 6.5    & 2.29  &            \\
%       七号  & 5.5    & 1.94  &            \\
%       八号  & 5      & 1.76  &            \\
%       \bottomrule
%     \end{tabular}
%   }
%   \caption{中文字号对应的字体大小}
%   \label{table:fontsize}
% \end{table}
%
% 下面定义中文字号对应的大小,其标准参见表\ref{table:fontsize:standard}和
% 表\ref{table:fontsize:word}。
%    \begin{macrocode}
\newcommand*{\zhbook@fs@eight}{5.02} % 八号字 5bp
\newcommand*{\zhbook@fs@eightskip}{6.02}
\newcommand*{\zhbook@fs@seven}{5.52} % 七号字 5.5bp
\newcommand*{\zhbook@fs@sevenskip}{6.62}
\newcommand*{\zhbook@fs@ssix}{6.52} % 小六号 6.5bp
\newcommand*{\zhbook@fs@ssixskip}{7.83}
\newcommand*{\zhbook@fs@six}{7.53} % 六号字 7.5bp
\newcommand*{\zhbook@fs@sixskip}{9.03}
\newcommand*{\zhbook@fs@sfive}{9.03} % 小五号 9bp
\newcommand*{\zhbook@fs@sfiveskip}{10.84}
\newcommand*{\zhbook@fs@five}{10.54} % 五号 10bp
\newcommand*{\zhbook@fs@fiveskip}{12.65}
\newcommand*{\zhbook@fs@sfour}{12.05} % 小四号 12bp
\newcommand*{\zhbook@fs@sfourskip}{14.45}
\newcommand*{\zhbook@fs@four}{14.05} % 四号字 14bp
\newcommand*{\zhbook@fs@fourskip}{16.86}
\newcommand*{\zhbook@fs@sthree}{15.06} % 小三号 15bp
\newcommand*{\zhbook@fs@sthreeskip}{18.07}
\newcommand*{\zhbook@fs@three}{16.06} % 三号字 16bp
\newcommand*{\zhbook@fs@threeskip}{19.27}
\newcommand*{\zhbook@fs@stwo}{18.07} % 小二号 18bp
\newcommand*{\zhbook@fs@stwoskip}{21.68}
\newcommand*{\zhbook@fs@two}{22.08} % 二号字 22bp
\newcommand*{\zhbook@fs@twoskip}{26.50}
\newcommand*{\zhbook@fs@sone}{24.09} % 小一号 24bp
\newcommand*{\zhbook@fs@soneskip}{28.91}
\newcommand*{\zhbook@fs@one}{26.10} % 一号字 26bp
\newcommand*{\zhbook@fs@oneskip}{31.32}
\newcommand*{\zhbook@fs@szero}{36.14} % 小初号 36bp
\newcommand*{\zhbook@fs@szeroskip}{43.36}
\newcommand*{\zhbook@fs@zero}{42.16} % 初号字 42bp
\newcommand*{\zhbook@fs@zeroskip}{50.59}
%    \end{macrocode}
%
% 声明不同字号下的数学字体大小。
%    \begin{macrocode}
\DeclareMathSizes{\zhbook@fs@eight}
                 {\zhbook@fs@eight}
                 {5}
                 {5}
\DeclareMathSizes{\zhbook@fs@seven}
                 {\zhbook@fs@seven}
                 {5}
                 {5}
\DeclareMathSizes{\zhbook@fs@ssix}
                 {\zhbook@fs@ssix}
                 {5}
                 {5}
\DeclareMathSizes{\zhbook@fs@six}
                 {\zhbook@fs@six}
                 {5}
                 {5}
\DeclareMathSizes{\zhbook@fs@sfive}
                 {\zhbook@fs@sfive}
                 {6}
                 {5}
\DeclareMathSizes{\zhbook@fs@five}
                 {\zhbook@fs@five}
                 {7}
                 {5}
\DeclareMathSizes{\zhbook@fs@sfour}
                 {\zhbook@fs@sfour}
                 {8}
                 {6}
\DeclareMathSizes{\zhbook@fs@four}
                 {\zhbook@fs@four}
                 {\zhbook@fs@five}
                 {\zhbook@fs@six}
\DeclareMathSizes{\zhbook@fs@sthree}
                 {\zhbook@fs@sthree}
                 {\zhbook@fs@sfour}
                 {\zhbook@fs@sfive}
\DeclareMathSizes{\zhbook@fs@three}
                 {\zhbook@fs@three}
                 {\zhbook@fs@four}
                 {\zhbook@fs@five}
\DeclareMathSizes{\zhbook@fs@stwo}
                 {\zhbook@fs@stwo}
                 {\zhbook@fs@sthree}
                 {\zhbook@fs@sfour}
\DeclareMathSizes{\zhbook@fs@two}
                 {\zhbook@fs@two}
                 {\zhbook@fs@three}
                 {\zhbook@fs@four}
\DeclareMathSizes{\zhbook@fs@sone}
                 {\zhbook@fs@sone}
                 {\zhbook@fs@stwo}
                 {\zhbook@fs@sthree}
\DeclareMathSizes{\zhbook@fs@one}
                 {\zhbook@fs@one}
                 {\zhbook@fs@two}
                 {\zhbook@fs@three}
\DeclareMathSizes{\zhbook@fs@szero}
                 {\zhbook@fs@szero}
                 {\zhbook@fs@sone}
                 {\zhbook@fs@stwo}
\DeclareMathSizes{\zhbook@fs@zero}
                 {\zhbook@fs@zero}
                 {\zhbook@fs@one}
                 {\zhbook@fs@two}
%    \end{macrocode}
%
% 定义字号选择命令。字号前面加负号表示采用对应的小体字号,例如|\zihao{-3}|表示小
% 三号。
% \begin{note}
% 为了让|\zihao{-0}|能正确表示小初号,在判断参数正负的时候把参数后面再接一个字符`1',从
% 而将``-0''变为``-01'',而``-01''转换为数字为$-1$,故可正确判断其是否小于零。
% \end{note}
%    \begin{macrocode}
\def\zhbook@zihao{}
\DeclareRobustCommand*{\zihao}[1]{%
  \def\zhbook@zihao{#1}%
  \ifnum #11<0%
    \@tempcnta=-#1
    \ifcase\@tempcnta%
        \fontsize\zhbook@fs@szero\zhbook@fs@szeroskip%
    \or \fontsize\zhbook@fs@sone\zhbook@fs@soneskip%
    \or \fontsize\zhbook@fs@stwo\zhbook@fs@stwoskip%
    \or \fontsize\zhbook@fs@sthree\zhbook@fs@sthreeskip%
    \or \fontsize\zhbook@fs@sfour\zhbook@fs@sfourskip%
    \or \fontsize\zhbook@fs@sfive\zhbook@fs@sfiveskip%
    \or \fontsize\zhbook@fs@ssix\zhbook@fs@ssixskip%
    \else \ClassError{zhbook}{%
            Undefined Chinese font size in command \protect\zihao}{%
            The old font size is used if you continue.}%
    \fi%
  \else%
    \@tempcnta=#1
    \ifcase\@tempcnta%
        \fontsize\zhbook@fs@zero\zhbook@fs@zeroskip%
    \or \fontsize\zhbook@fs@one\zhbook@fs@oneskip%
    \or \fontsize\zhbook@fs@two\zhbook@fs@twoskip%
    \or \fontsize\zhbook@fs@three\zhbook@fs@threeskip%
    \or \fontsize\zhbook@fs@four\zhbook@fs@fourskip%
    \or \fontsize\zhbook@fs@five\zhbook@fs@fiveskip%
    \or \fontsize\zhbook@fs@six\zhbook@fs@sixskip%
    \or \fontsize\zhbook@fs@seven\zhbook@fs@sevenskip%
    \or \fontsize\zhbook@fs@eight\zhbook@fs@eightskip%
    \else \ClassError{zhbook}{%
            Undefined Chinese font size in command \protect\zihao}{%
            The old font size is used if you continue.}%
    \fi%
  \fi%
  \selectfont\ignorespaces}
%    \end{macrocode}
%
% 修改常用字体大小选择命令。
%
%    \begin{macrocode}
\renewcommand{\tiny}{% 小六号 6.5bp
  \@setfontsize\tiny{\zhbook@fs@ssix}{\zhbook@fs@ssixskip}}
\renewcommand{\scriptsize}{% 六号字 7.5bp
  \@setfontsize\scriptsize{\zhbook@fs@six}{\zhbook@fs@sixskip}}
\renewcommand{\footnotesize}{% 小五号 9bp
  \@setfontsize\footnotesize{\zhbook@fs@sfive}{\zhbook@fs@sfiveskip}%
  \abovedisplayskip 6\p@ \@plus2\p@ \@minus4\p@
  \abovedisplayshortskip \z@ \@plus\p@
  \belowdisplayshortskip 3\p@ \@plus\p@ \@minus2\p@
  \def\@listi{\leftmargin\leftmargini
    \topsep 3\p@ \@plus\p@ \@minus\p@
    \parsep 2\p@ \@plus\p@ \@minus\p@
    \itemsep \parsep}%
  \belowdisplayskip \abovedisplayskip}
\renewcommand{\small}{% 五号 10bp
  \@setfontsize\small{\zhbook@fs@five}{\zhbook@fs@fiveskip}%
  \abovedisplayskip 8.5\p@ \@plus3\p@ \@minus4\p@
  \abovedisplayshortskip \z@ \@plus2\p@
  \belowdisplayshortskip 4\p@ \@plus2\p@ \@minus2\p@
  \def\@listi{\leftmargin\leftmargini
    \topsep 4\p@ \@plus2\p@ \@minus2\p@
    \parsep 2\p@ \@plus\p@ \@minus\p@
    \itemsep \parsep}%
  \belowdisplayskip \abovedisplayskip}
\renewcommand{\normalsize}{% 小四号 12bp
  \@setfontsize\normalsize{\zhbook@fs@sfour}{\zhbook@fs@sfourskip}%
  \abovedisplayskip 10\p@ \@plus2\p@ \@minus5\p@
  \abovedisplayshortskip \z@ \@plus3\p@
  \belowdisplayshortskip 6\p@ \@plus3\p@ \@minus3\p@
  \belowdisplayskip \abovedisplayskip
  \let\@listi\@listI}
\renewcommand{\large}{% 小三号 15bp
  \@setfontsize\large{\zhbook@fs@sthree}{\zhbook@fs@sthreeskip}}
\renewcommand{\Large}{% 小二号 18bp
  \@setfontsize\Large{\zhbook@fs@stwo}{\zhbook@fs@stwoskip}}
\renewcommand{\LARGE}{% 小一号 24bp
  \@setfontsize\LARGE{\zhbook@fs@sone}{\zhbook@fs@soneskip}}
\renewcommand{\huge}{% 一号 26bp
  \@setfontsize\huge{\zhbook@fs@one}{\zhbook@fs@oneskip}}
\renewcommand{\Huge}{% 小初号 36bp
  \@setfontsize\Huge{\zhbook@fs@szero}{\zhbook@fs@szeroskip}}
%    \end{macrocode}
%
% 定义中文字距修改命令,直接修改\cs{CJKglue}即可。
%    \begin{macrocode}
\newcommand*{\ziju}[1]{\renewcommand*{\CJKglue}{\hskip {#1}}}
%    \end{macrocode}
%
% 修改\cs{textsc}命令,使其可在中文编码下正常工作。
%    \begin{macrocode}
\renewcommand{\textsc}[1]{{\usefont{OT1}{cmr}{m}{sc}{#1}}}
%    \end{macrocode}
%
% \subsection{数学公式和定理}
%
% 按照\std{CY/T 35-2001}规范的要求,重定义公式、图、表的编号格式。例如:
% \begin{itemize}
% \item 图\dashnumber{1}{2}
% \item 表\dashnumber{2}{3}
% \item 附注 1)
% \item 文献[4]
% \item 式(\dashnumber{6}{3})
% \end{itemize}
% 子图和子表的应用序号外加小括号,例如
% \begin{itemize}
% \item 图\dashnumber{1}{2}(a)
% \item 表\dashnumber{2}{3}(b)
% \end{itemize}
%    \begin{macrocode}
\newcommand{\dashnumber}[2]%
  {{#1}\kern.07em\rule[.5ex]{.4em}{.15ex}\kern.07em{#2}}
\renewcommand*{\thefigure}{\dashnumber{\thechapter}{\arabic{figure}}}
\renewcommand*{\thetable}{\dashnumber{\thechapter}{\arabic{table}}}
\renewcommand*{\theequation}{\dashnumber{\thechapter}{\arabic{equation}}}
\renewcommand*{\thesubfigure}{(\alph{subfigure})}
\renewcommand*{\thesubtable}{(\alph{subtable})}
%    \end{macrocode}
%
% 定义常用的数学定理环境及其样式。
%    \begin{macrocode}
\newtheoremstyle{plain}% name
                {1em}%      Space above, empty = `usual value'
                {1em}%      Space below
                {\normalfont}% Body font
                {}%         Indent amount
                {\normalfont\bfseries}% Thm head font
                {}%         Punctuation after thm head
                {1em}%      Space after thm head: \newline = linebreak
                {}%         Thm head spec
\newtheorem{definition}{\zhbook@cap@definition}[chapter]
\newtheorem{notation}[definition]{\zhbook@cap@notation}
\newtheorem{theorem}{\zhbook@cap@theorem}[chapter]
\newtheorem{lemma}[theorem]{\zhbook@cap@lemma}
\newtheorem{corollary}[theorem]{\zhbook@cap@corollary}
\newtheorem{proposition}[theorem]{\zhbook@cap@proposition}
\newtheorem{fact}[theorem]{\zhbook@cap@fact}
\newtheorem{assumption}[theorem]{\zhbook@cap@assumption}
\newtheorem{conjecture}[theorem]{\zhbook@cap@conjecture}
\newtheorem{hypothesis}{\zhbook@cap@hypothesis}[chapter]
\newtheorem{axiom}{\zhbook@cap@axiom}[chapter]
\newtheorem{postulate}{\zhbook@cap@postulate}[chapter]
\newtheorem{principle}{\zhbook@cap@principle}[chapter]
\newtheorem{problem}{\zhbook@cap@problem}[chapter]
\newtheorem{exercise}{\zhbook@cap@exercise}[chapter]
\newtheorem{example}{\zhbook@cap@example}[chapter]
\newtheorem{remark}{\zhbook@cap@remark}[chapter]
\newtheorem{result}{\zhbook@cap@result}[chapter]

\renewenvironment{proof}[1][\zhbook@cap@proof]{\par
  \pushQED{\qed}%
  \normalfont \topsep6\p@\@plus6\p@\relax
  \trivlist
  \item[\hskip\labelsep\textbf{#1}\@addpunct{:}]\ignorespaces
}{\popQED\endtrivlist\@endpefalse}

\newenvironment{solution}[1][\zhbook@cap@solution]{\par
  \normalfont \topsep6\p@\@plus6\p@\relax
  \trivlist
  \item[\hskip\labelsep\textbf{#1}\@addpunct{:}]\ignorespaces
}{\endtrivlist\@endpefalse}

\newtheorem{algorithm}{\zhbook@cap@algorithm}[chapter]
%    \end{macrocode}
%
% 修改上面定义的各定理环境的编号样式:
%    \begin{macrocode}
\renewcommand*{\thedefinition}{\dashnumber{\thechapter}{\arabic{definition}}}
\renewcommand*{\thetheorem}{\dashnumber{\thechapter}{\arabic{theorem}}}
\renewcommand*{\theaxiom}{\dashnumber{\thechapter}{\arabic{axiom}}}
\renewcommand*{\theproblem}{\dashnumber{\thechapter}{\arabic{problem}}}
\renewcommand*{\theexercise}{\dashnumber{\thechapter}{\arabic{exercise}}}
\renewcommand*{\theexample}{\dashnumber{\thechapter}{\arabic{example}}}
\renewcommand*{\theremark}{\dashnumber{\thechapter}{\arabic{remark}}}
%    \end{macrocode}
%
% \subsection{设置浮动环境格式}
%
% 默认情况下,{\LaTeX}要求每页的文字至少占据$20\%$,否则该页就只单独放置一个浮动环境。而
% 这通常不是我们想要的。我们将这个要求降低到$5\%$。
%    \begin{macrocode}
\renewcommand*{\textfraction}{0.05}
%    \end{macrocode}
% 有时如果多个浮动环境连续放在一起,{\LaTeX}会将它们分在几个不同页,即使它们可在同一页放
% 得下。我们可以通过修改\cs{topfraction}和\cs{bottomfraction}分别设置顶端和底端的浮动
% 环境的最大比例。
%    \begin{macrocode}
\renewcommand*{\topfraction}{0.9}
\renewcommand*{\bottomfraction}{0.8}
%    \end{macrocode}
% 有时{\LaTeX}会把一个浮动环境单独放在一页,我们要求这个环境至少要占据$85\%$才能单独放在
% 一页。
% \begin{note}
% \cs{floatpagefraction}的数值必须小于\cs{topfraction}。
% \end{note}
%    \begin{macrocode}
\renewcommand*{\floatpagefraction}{0.85}
%    \end{macrocode}
%
% \subsection{中文标题名称}
%
% 设置常见的中文标题名称。
%    \begin{macrocode}
\newcommand*{\abstractname}{\zhbook@cap@abstract}
\renewcommand*{\contentsname}{\zhbook@cap@contents}
\renewcommand*{\listfigurename}{\zhbook@cap@listfigure}
\renewcommand*{\listtablename}{\zhbook@cap@listtable}
\newcommand*{\listsymbolname}{\zhbook@cap@listsymbol}
\newcommand*{\listequationname}{\zhbook@cap@listequation}
\renewcommand*{\glossaryname}{\zhbook@cap@glossary}
\renewcommand*{\indexname}{\zhbook@cap@index}
\newcommand*{\equationname}{\zhbook@cap@equation}
\renewcommand*{\bibname}{\zhbook@cap@bib}
\renewcommand*{\figurename}{\zhbook@cap@figure}
\renewcommand*{\tablename}{\zhbook@cap@table}
\renewcommand*{\chaptername}{\zhbook@cap@chapter}
\renewcommand*{\appendixname}{\zhbook@cap@appendix}
%    \end{macrocode}
%
% \subsection{中文标题格式}
%
% 设置章节格式如下:
% \begin{description}
% \item[零级节标题] 命令为\cs{chapter},格式为一号黑体,居中排列,段前空4ex,段后空3ex;
% \item[一级节标题] 命令为\cs{section},格式为小二号黑体,左排列,段前空3.5ex,段后空2.3ex;
% \item[二级节标题] 命令为\cs{subsection},格式为三号黑体,左排列,段前空3.0ex,段后空1.5ex;
% \item[三级节标题] 命令为\cs{subsubsection},格式为小三号黑体,左排列,段前空2.5ex,段后空1.5ex;
% \item[四级节标题] 命令为\cs{paragraph},格式为四号黑体,左排列,段前空2.0ex,段后空1ex;
% \item[五级节标题] 命令为\cs{subparagraph},格式为小四号黑体,左排列,段前空1.5ex,段后空1ex;
% \end{description}
%
% 使用|titlesec|宏包提供的\cs{titleformat}和\cs{titlespacing}命令可以方便地设置标题
% 的样式:
%    \begin{macrocode}
\titleformat{\chapter}[hang]
            {\centering\zihao{-1}\bfseries}
            {\chaptertitlename}{1em}{}
\titlespacing{\chapter}
             {0pt}
             {*4}
             {*3}
\titleformat{\section}[hang]
            {\zihao{-2}\bfseries}
            {\thesection}{1em}{}
\titlespacing{\section}
             {0pt}
             {*3.5}
             {*2.3}
\titleformat{\subsection}[hang]
            {\zihao{3}\bfseries}
            {\thesubsection}{1em}{}
\titlespacing{\subsection}
             {0pt}
             {*3}
             {*1.5}
\titleformat{\subsubsection}[hang]
            {\zihao{-3}\bfseries}
            {\thesubsubsection}{1em}{}
\titlespacing{\subsubsection}
             {0pt}
             {*2.5}
             {*1.5}
\titleformat{\paragraph}[hang]
            {\zihao{4}\bfseries}
            {}{0em}{}
\titlespacing{\paragraph}
             {0pt}
             {*2}
             {*1}
\titleformat{\subparagraph}[hang]
            {\zihao{-4}\bfseries}
            {}{0em}{}
\titlespacing{\subparagraph}
             {0pt}
             {*1.5}
             {*1}
%    \end{macrocode}
%
% 设置章节标题编号最多到第4层(即\cs{subsubsection}),超过第四层的章节不再自动编号。
%    \begin{macrocode}
\setcounter{secnumdepth}{4}
%    \end{macrocode}
%
% 修改章节编号的样式:
%    \begin{macrocode}
\renewcommand{\thechapter}{\arabic{chapter}}
\renewcommand{\thesection}{\thechapter\thinspace.\thinspace\arabic{section}}
\renewcommand{\thesubsection}{\thesection\thinspace.\thinspace\arabic{subsection}}
\renewcommand{\thesubsubsection}{\thesubsection\thinspace.\thinspace\arabic{subsubsection}}
%    \end{macrocode}
%
% \subsection{浮动环境}
%
% 设置浮动环境标题的字体大小。根据学位书籍格式要求,插图和表格标题字体需要比正文字体略小。
%    \begin{macrocode}
\captionsetup{font=small}
%    \end{macrocode}
%
% 根据学位书籍格式要求,表格的标题必须位于表格上方,插图的标题必须位于插图下方。
%    \begin{macrocode}
\captionsetup[table]{position=above}
\captionsetup[figure]{position=below}
\floatstyle{plaintop}
\restylefloat{table}
%    \end{macrocode}
%
% \subsection{页幅设置}
%
% 正文统一用小四号字,间距为固定值20pt。\cs{linestrech}的值为$1$时为单倍行距, $1.2$时是
% 一倍半行距, 而为$1.6$时是双倍行距。其实不同尺寸的字体行间距都不相同,而是成比例关系。这
% 个20pt是对正文主要字体来说的。
%
% 在{\TeX}中基本的行间距是\cs{baselineskip}, 对于12pt的字体,这个值等于14.5pt,
% 而真正的行间距是\cs{baselineskip}$\times$\cs{baselinestretch},
% \cs{baselinestretch}默认为$1$, 但我们可以重新设置它的值,如
% |\renewcommand{\baselinestretch}{1.38}|就得到真正的行间距为14.5pt*1.38≈20pt。
% 而这样定义之后,对不同尺寸的字体都会按同样的比例因子1.38放大行间距,使得全文排
% 版能协调一致。
%    \begin{macrocode}
\renewcommand*{\baselinestretch}{1.38}
%    \end{macrocode}
%
% 修改|tabular|环境,设置表格中的行间距为正文行间距。
%    \begin{macrocode}
\let\zhbook@oldtabular\tabular
\let\zhbook@endoldtabular\endtabular
\renewenvironment{tabular}%
{\bgroup%
\renewcommand{\arraystretch}{1.38}%
\zhbook@oldtabular}%
{\zhbook@endoldtabular\egroup}
%    \end{macrocode}
%
% 文章用A4纸标准大小的白纸打印,页眉:2.6cm,页脚:2.4cm,页边距上下:3.5cm,左
% 右:3.2cm。
%    \begin{macrocode}
\geometry{headheight=2.6cm,headsep=3mm,footskip=13mm}
\geometry{top=3.5cm,bottom=3.5cm,left=3.2cm,right=3.2cm}
%    \end{macrocode}
%
% 设置每一段的首行缩进两个汉字。
% \begin{note}
% 直接将|parindent|设置为|2em|并不能正确地设置段首缩进为恰好两个中文字符。因此我们采用下面的
% 网页提供办法:\\
% \url{https://github.com/ElegantLaTeX/ElegantLaTeX/blob/master/CJKindent.md}
% \end{note}
%
% 首先,我们需要计算出当前字符的宽度:
%    \begin{macrocode}
\def\zhbook@CJK@charwidth{\hskip \f@size \p@}
%    \end{macrocode}
% 接下来,我们需要考虑到字符间距,计算出当前相邻两字符中心的距离:
%    \begin{macrocode}
\newdimen\zhbook@CJK@chardimen
\settowidth\zhbook@CJK@chardimen{\zhbook@CJK@charwidth\CJKglue}
%    \end{macrocode}
% 最后,我们设置段首缩进长度:
%    \begin{macrocode}
\newcommand{\zhbook@CJK@setfontspace}{%
\settowidth\zhbook@CJK@chardimen{\zhbook@CJK@charwidth\CJKglue}%
\ifdim\parindent=0pt\relax\else\parindent2\zhbook@CJK@chardimen\fi%
}
\renewcommand*{\indent}{\zhbook@CJK@setfontspace\parindent2\zhbook@CJK@chardimen}
\AtBeginDocument{\indent}
%    \end{macrocode}
%
% \subsection{页眉页脚}
%
% 我们使用|fancyhdr|宏包来设置页眉页脚。|fancyhdr|宏包提供了一个|fancy|页面风格,
% 在该风格下,章节的起始页(即包含“第XX章”标题的页面)的页眉页脚将使用|plain|风
% 格,而章节的后继页面的页眉页脚将使用|fancy|风格的默认定义或用户通过
% \cs{fancyhead}或\cs{fancyfoot}命令定义的样式。
%
% 首先,我们需要修改|plain|风格的页眉页脚,将其页脚默认的页码去掉。
%    \begin{macrocode}
\fancypagestyle{plain}{%
   \fancyhead{}                       % get rid of headers and footers
   \renewcommand{\headrulewidth}{0pt} % and the header line
   \renewcommand{\footrulewidth}{0pt} % and the footer line
}
%    \end{macrocode}
%
% 接下来我们按照如下规则修改|fancy|风格的页眉页脚设置,注意学位书籍始终是双面打印的:
% \begin{itemize}
%    \item 令偶数页的页眉如下:
%      \begin{itemize}
%      \item 左上角显示当前页页码
%      \item 右上角显示当前章(chapter)的编号和标题;
%      \item 若当前不为于|mainmatter|中,则右上角只显示当前章的标题。
%      \end{itemize}
%    \item 令奇数页的页眉如下:
%      \begin{itemize}
%      \item 左上角显示当前节(section)的编号和标题
%      \item 右上角显示当前页页码;
%      \item 若当前页面尚未开始此章的第一节,即节编号和节标题为空;则左上角显示当前
%      章(chapter)的编号和标题;若当前不为于|mainmatter|中,则左上角只显示当前章的
%      标题。
%      \end{itemize}
%    \item 无论奇偶页,页眉下都有一条分割线;
%    \item 无论奇偶页,页脚都为空,页脚上都无分割线。
% \end{itemize}
%
% 设置|fancy|风格下的页脚,令页脚为空;令页脚分割线宽度为$0$:
%    \begin{macrocode}
\fancyfoot{}
\renewcommand{\footrulewidth}{0pt}
%    \end{macrocode}
%
% 设置|fancy|风格下的页眉,令偶数页左上角和奇数页右上角显示当前页码,令页眉的分
% 割线宽度为$1$:
%    \begin{macrocode}
\fancyhead[LE,RO]{\thepage}
\renewcommand{\headrulewidth}{1pt}
%    \end{macrocode}
%
% 设置|fancy|风格的页眉,令偶数页右上角和奇数页左上角分别显示当前章信息和当前节
% 信息;但若当前页面尚未开始本章的第一节(即\cs{rightmark}为空),则奇数页左上角也
% 显示当前章信息(即\cs{leftmark})。
%    \begin{macrocode}
\fancyhead[RE]{\leftmark}
\fancyhead[LO]{%
 \ifthenelse{\equal{\rightmark}{}}% if \rightmark is empty
            {\leftmark}%
            {\rightmark}%
}
%    \end{macrocode}
%
% 设置全局使用|fancy|风格。
%    \begin{macrocode}
\pagestyle{fancy}
%    \end{macrocode}
%
% 重新定义|chaptermark|,让其显示当前章信息和当前节信息。注意下面的重定义必须放
% 在第一次调用|\pagestyle{fancy}|之后,因为第一次调用该命令会设置\cs{chaptermark}。
%    \begin{macrocode}
\renewcommand{\chaptermark}[1]{\markboth{%
 \bfseries\if@mainmatter\chaptertitlename\hspace{1em}\fi{#1}%
}{}}
%    \end{macrocode}
%
% 重新定义|sectionmark|,让其显示当前节信息。注意下面的重定义必须放在第一次调用
% |\pagestyle{fancy}|之后,因为第一次调用该命令会设置\cs{sectionmark}。
%    \begin{macrocode}
\renewcommand{\sectionmark}[1]{\markright{%
 \bfseries\if@mainmatter\thesection\hspace{1em}\fi{#1}%
}}
%    \end{macrocode}
%
% 另一个麻烦的问题是:默认的|fancy|风格会在每一章最后的空白页(由于是双面打印)
% 也加上页眉页脚,但我们通常不希望如此。解决方法是修改{\LaTeX}内部的\cs{cleardoublepage}
% 命令的定义如下:
%    \begin{macrocode}
\def\cleardoublepage{\clearpage\if@twoside \ifodd\c@page\else
  \hbox{}\thispagestyle{empty}\newpage\if@twocolumn\hbox{}\newpage\fi\fi\fi}
%    \end{macrocode}
%
% \subsection{列表环境}
%
% {\LaTeX}默认的列表:|enumerate|,|itemize|,和|description|都不符合中文习惯。
% 符合中文习惯的列表需要满足:
% \begin{enumerate}
% \item 列表标签要与正文的左边界对齐;
% \item 列表文本左侧要和左边界对齐;
% \item 列表项的间距应当等于正文中的段落间距,通常为$0$;
% \item 列表文本的右侧与正文的右边界对齐。
% \end{enumerate}
% 因此需要重新设置默认的列表的格式。
%    \begin{macrocode}
\setlist{%
  topsep=0.3em,             % 列表顶端的垂直空白
  partopsep=0pt,            % 列表环境前面紧接着一个空白行时其顶端的额外垂直空白
  itemsep=0ex plus 0.1ex,   % 列表项之间的额外垂直空白
  parsep=0pt,               % 列表项内的段落之间的垂直空白
  leftmargin=1.5em,         % 环境的左边界和列表之间的水平距离
  rightmargin=0em,          % 环境的右边界和列表之间的水平距离
  labelsep=0.5em,           % 包含标签的盒子与列表项的第一行文本之间的间隔
  labelwidth=2em,           % 包含标签的盒子的正常宽度;若实际宽度更宽,则使用实际宽度。
}
%    \end{macrocode}
%
% 设置无序列表的标签符号。
%    \begin{macrocode}
\setlist[itemize,1]{label=$\medbullet$}
\setlist[itemize,2]{label=$\blacksquare$}
\setlist[itemize,3]{label=$\Diamondblack$}
%    \end{macrocode}
%
% \subsection{引用}
%
% 默认的引用环境|quote|和|quotation|都不符合中文习惯,我们将其重新定义如下:
%    \begin{macrocode}
\renewenvironment{quote}%
                 {\list{}{\leftmargin=4em\rightmargin=4em}\item[]}%
                 {\endlist}
\renewenvironment{quotation}%
                 {\list{}{\leftmargin=4em\rightmargin=4em}\item[]}%
                 {\endlist}
%    \end{macrocode}
%
% \subsection{目次}
%
% 前置部分的封面在后面详细介绍,首先看目次。其具体要求为:目次页由书籍的章、节、条、项、
% 附录等的序号、名称和页码组成,另页排在序之后。目次页标注学位书籍的前三级目录。
% 标题统一用“目次”,黑体3字号字居中,段前、段后间距为1行; 各章(一级目录)名称用
% 黑体5号字,段前间距为0.5行,段后间距为0行; 其它(二、三级目录)用宋体5号字,
% 段前、段后间距为0行。
%
% \begin{macro}{\nchapter}
% 用于产生没有编号但在目次中列出的章。
%    \begin{macrocode}
\newcommand\nchapter[1]{%
  \if@mainmatter%
    \@mainmatterfalse%
    \chapter{#1}%
    \@mainmattertrue%
  \else
    \chapter{#1}%
  \fi
}
%    \end{macrocode}
% \end{macro}
%
% \begin{macro}{\@dottedtocline}
% 改变缺省的目次中的点线为中文习惯。
%    \begin{macrocode}
\def\@dottedtocline#1#2#3#4#5{%
  \ifnum #1>\c@tocdepth \else
    \vskip \z@ \@plus.2\p@
    {\leftskip #2\relax \rightskip \@tocrmarg \parfillskip -\rightskip
     \parindent #2\relax\@afterindenttrue
     \interlinepenalty\@M
     \leavevmode
     \@tempdima #3\relax
     \advance\leftskip \@tempdima \null\nobreak\hskip -\leftskip
     {#4}\nobreak
     \leaders\hbox{$\m@th\mkern 1.5mu\cdot\mkern 1.5mu$}\hfill
     \nobreak
     \hb@xt@\@pnumwidth{\hfil\normalfont \normalcolor #5}%
     \par}%
  \fi}
%    \end{macrocode}
% \end{macro}
%
% \begin{macro}{\l@part}
% 改变缺省的目次中的点线为中文习惯。
%    \begin{macrocode}
\renewcommand*{\l@part}[2]{%
  \ifnum \c@tocdepth >-2\relax
    \addpenalty{-\@highpenalty}%
    \addvspace{2.25em \@plus\p@}%
    \setlength\@tempdima{3em}%
    \begingroup
      \parindent \z@ \rightskip \@pnumwidth
      \parfillskip -\@pnumwidth
      {\leavevmode
       \large \bfseries #1
       \leaders\hbox{$\m@th\mkern 1.5mu\cdot\mkern 1.5mu$}
       \hfil \hb@xt@\@pnumwidth{\hss #2}}\par
       \nobreak
         \global\@nobreaktrue
         \everypar{\global\@nobreakfalse\everypar{}}%
    \endgroup
  \fi}
%    \end{macrocode}
% \end{macro}
%
% \begin{macro}{\l@chapter}
% 改变缺省的目次中的点线为中文习惯。
%    \begin{macrocode}
\renewcommand*{\l@chapter}[2]{%
  \ifnum \c@tocdepth >\m@ne
    \addpenalty{-\@highpenalty}%
    \vskip 1.0em \@plus\p@
    \setlength\@tempdima{1.5em}%
    \begingroup
      \parindent \z@ \rightskip \@pnumwidth
      \parfillskip -\@pnumwidth
      \leavevmode \bfseries
      \advance\leftskip\@tempdima
      \hskip -\leftskip
      #1\nobreak
      \leaders\hbox{$\m@th\mkern 1.5mu\cdot\mkern 1.5mu$}
      \hfil \nobreak\hb@xt@\@pnumwidth{\hss #2}\par
      \penalty\@highpenalty
    \endgroup
  \fi}
%    \end{macrocode}
% \end{macro}
%
% \begin{macro}{\tableofcontents}
% 修改\cs{tableofcontents}命令用于生成目次页,并将目次页本身也被加入目次中。
%    \begin{macrocode}
\renewcommand*{\tableofcontents}{%
    \if@twocolumn
      \@restonecoltrue\onecolumn
    \else
      \@restonecolfalse
    \fi
    \nchapter{\contentsname}%
    \@mkboth{\MakeUppercase\contentsname}{\MakeUppercase\contentsname}%
    \@starttoc{toc}%
    \if@restonecol\twocolumn\fi
}
%    \end{macrocode}
% \end{macro}
%
% \begin{macro}{\listoftables}
% 修改\cs{listoftables}命令,使得附表清单被加入目次中。
%    \begin{macrocode}
\renewcommand*{\listoftables}{%
    \if@twocolumn
      \@restonecoltrue\onecolumn
    \else
      \@restonecolfalse
    \fi
    \nchapter{\listtablename}%
    \@mkboth{\MakeUppercase\listtablename}{\MakeUppercase\listtablename}%
    \@starttoc{lot}%
    \if@restonecol\twocolumn\fi
}
%    \end{macrocode}
% \end{macro}
%
% \begin{macro}{\listoffigures}
% 修改\cs{listoffigures}命令,使得插图清单被加入目次中。
%    \begin{macrocode}
\renewcommand*{\listoffigures}{%
    \if@twocolumn
      \@restonecoltrue\onecolumn
    \else
      \@restonecolfalse
    \fi
    \nchapter{\listfigurename}%
    \@mkboth{\MakeUppercase\listfigurename}{\MakeUppercase\listfigurename}%
    \@starttoc{lof}%
    \if@restonecol\twocolumn\fi
}
%    \end{macrocode}
% \end{macro}
%
% \subsection{参考文献}
%
% \begin{environment}{thebibliography}
% 修改|thebibliography|环境用于在目次中加入参考文献页。
%    \begin{macrocode}
\renewenvironment{thebibliography}[1]
     {\nchapter{\bibname}%
      \@mkboth{\MakeUppercase\bibname}{\MakeUppercase\bibname}%
      \list{\@biblabel{\@arabic\c@enumiv}}%
           {\settowidth\labelwidth{\@biblabel{#1}}%
            \leftmargin\labelwidth
            \advance\leftmargin\labelsep
            \@openbib@code
            \usecounter{enumiv}%
            \let\p@enumiv\@empty
            \renewcommand\theenumiv{\@arabic\c@enumiv}}%
      \sloppy
      \clubpenalty4000
      \@clubpenalty \clubpenalty
      \widowpenalty4000%
      \sfcode`\.\@m}
     {\def\@noitemerr
       {\@latex@warning{Empty `thebibliography' environment}}%
      \endlist}
%    \end{macrocode}
% \end{environment}
%
% 使用|gbt7714-2005.bst|作为参考文献样式。
%    \begin{macrocode}
\bibliographystyle{gbt7714-2005}
%    \end{macrocode}
%
% 使用符合\std{GB/T 7714-2005}规范的参考文献引用样式。
%    \begin{macrocode}
\setcitestyle{super,square}
%    \end{macrocode}
%
% 修改|natbib|内部的\cs{NAT@citesuper}命令,使其生成的上标引用编号可以正确地把
% \cs{cite}命令的可选参数(通常是引文页码)也作为上标放在引文编号方框之后。
%    \begin{macrocode}
\renewcommand\NAT@citesuper[3]{%
\ifNAT@swa%
  \if*#2*\else#2\NAT@spacechar\fi%
  \unskip\kern\p@\textsuperscript{\NAT@@open#1\NAT@@close#3}%
\else #1\fi\endgroup%
}
%    \end{macrocode}
%
% \subsection{脚注}
%
% 使用|footmisc|宏包和|pifont|宏包设置符合\std{GB/T 7713.1-2006}规范的脚注样式。注意,
% 由于|pifont|宏包提供的特殊符号的限制,一页之中最多只能有$10$个脚注。
%    \begin{macrocode}
\DefineFNsymbols*{circlednumber}[text]{%
   {\ding{192}} %
   {\ding{193}} %
   {\ding{194}} %
   {\ding{195}} %
   {\ding{196}} %
   {\ding{197}} %
   {\ding{198}} %
   {\ding{199}} %
   {\ding{200}} %
   {\ding{201}} %
}%
\setfnsymbol{circlednumber}
%    \end{macrocode}
%
% \subsection{封面字段设置}
%
% 定义默认封面字段值
%    \begin{macrocode}
\newcommand*{\zhbook@value@title}{(书籍标题)}
\newcommand*{\zhbook@value@en@title}{English Book Title}
\newcommand*{\zhbook@value@author}{(作者姓名)}
\newcommand*{\zhbook@value@abstract}{}
\newcommand*{\zhbook@value@abstract@keywords}{}
\newcommand*{\zhbook@value@publisherlogo}{}
\newcommand*{\zhbook@value@publisher}{出版社名称}
\newcommand*{\zhbook@value@publishercity}{北\hspace{1.5em}京}
\newcommand*{\zhbook@value@date}{{\CJKnumber\year}年}
%    \end{macrocode}
%
% 中文封面字段设置:
%    \begin{macrocode}
\renewcommand*{\title}[1]{\renewcommand*{\zhbook@value@title}{#1}}
\newcommand*{\titlea}[1]{\renewcommand*{\zhbook@value@titlea}{#1}}
\newcommand*{\titleb}[1]{\renewcommand*{\zhbook@value@titleb}{#1}}
\renewcommand*{\author}[1]{\renewcommand*{\zhbook@value@author}{#1}}
\newcommand*{\publisher}[1]{\renewcommand*{\zhbook@value@publisher}{#1}}
\newcommand*{\publishercity}[1]{\renewcommand*{\zhbook@value@publishercity}{#1}}
\newcommand*{\publisherlogo}[1]{\renewcommand*{\zhbook@value@publisherlogo}{#1}}
\renewcommand*{\date}[1]{\renewcommand*{\zhbook@value@date}{#1}}
\newcommand*{\abstract}[1]{\renewcommand*{\zhbook@value@abstract}{#1}}
\newcommand*{\keywords}[1]{\renewcommand*{\zhbook@value@abstract@keywords}{#1}}
%    \end{macrocode}
%
% 英文封面字段设置:
%    \begin{macrocode}
\newcommand*{\englishtitle}[1]{\renewcommand{\zhbook@value@en@title}{#1}}
%    \end{macrocode}
%
% \subsection{生成封面}
%
% \begin{macro}{\zhbookunderline}
% 定义封面中用到的生成下划线的宏。
%    \begin{macrocode}
\newcommand{\zhbook@underline}[2][\textwidth]%
           {\CJKunderline{\makebox[#1]{#2}}}
\def\zhbookunderline{\@ifnextchar[\zhbook@underline\CJKunderline}
%    \end{macrocode}
% \end{macro}
%
% \begin{macro}{\maketitle}
% 重新定义{\LaTeX}提供的\cs{maketitle}命令,使其生成中文科技书籍所需的中文封面。
% 注意我们使用了前面修改过的\cs{cleardoublepage}命令来插入无页眉页脚的空白页。
%    \begin{macrocode}
\renewcommand*{\maketitle}{%
  \thispagestyle{empty}
  \pdfbookmark[0]{\zhbook@cap@cover}{cover}
  \begin{center}
    \vspace{50mm}%
    \rule{\textwidth}{2pt}\\%
    \vspace{10mm}%
    {\zihao{1}\textbf{\zhbook@value@title}}\\
    \vspace{10mm}%
    {\zihao{-1}\textsf{\textbf{\zhbook@value@en@title}}}\\
    \vspace{6mm}%
    \rule{\textwidth}{2pt}\\%
    \vspace{10mm}
    {\zihao{-2}\textsl{\zhbook@value@author}}\\
    \vskip\stretch{1}%
    {\zihao{1}\textsl{\zhbook@value@publisher}}\\
    \vspace{6mm}%
    {\zihao{3}{\zhbook@value@publishercity}} \\
    \vspace{3mm}%
    {\zihao{3}{\zhbook@value@date}}
  \end{center}
  \clearpage
  \thispagestyle{empty}
  \begin{center}
    \LARGE\textbf{\zhbook@cap@abstract}
  \end{center}
  \vspace{4mm}%

  \clearpage
}
%    \end{macrocode}
%  \end{macro}
%
% \subsection{序言章节}
%
% \begin{environment}{prologue}
% 该环境用于``序言''页。
%    \begin{macrocode}
\newenvironment{prologue}{%
  \nchapter{\zhbook@cap@prologue}
}{}
%    \end{macrocode}
% \end{environment}
%
% \subsection{前言章节}
%
% \begin{environment}{preface}
% 该环境用于``前言''页。
%    \begin{macrocode}
\newenvironment{preface}{%
  \nchapter{\zhbook@cap@preface}
}{}
%    \end{macrocode}
% \end{environment}
%
% \subsection{致谢章节}
%
% \begin{environment}{acknowledgement}
% 该环境用于``致谢''页。
%    \begin{macrocode}
\newenvironment{acknowledgement}{%
  \nchapter{\zhbook@cap@acknowledgement}
}{}
%    \end{macrocode}
% \end{environment}
%
% \subsection{其他自定义命令和环境}
%
% \begin{macro}{\zhbook}
% 定义{\zhbook}文档类的logo。
%    \begin{macrocode}
\newcommand{\zhbook}{\texttt{NJU-Thesis}}
%    \end{macrocode}
% \end{macro}
%
% \begin{macro}{\zhdash}
% 定义中文破折号。
%    \begin{macrocode}
\newcommand{\zhdash}{\kern0.3ex\rule[0.8ex]{2em}{0.1ex}\kern0.3ex}
%    \end{macrocode}
% \end{macro}
%
% \begin{macro}{\cell}
% \cs{cell}\marg{width}\marg{height}\marg{text}用于定义一个宽度为\meta{width},
% 高度为\meta{height},内容为\meta{text}的的单元格。该单元格可放在表格中,用于控
% 制表格单元格的大小。
%    \begin{macrocode}
\newcommand{\cell}[3]{\parbox[c][#2][c]{#1}{\makebox[#1]{#3}}}
%    \end{macrocode}
% \end{macro}
%
% \begin{macro}{C}
% 定义一个新的表格列模式,|C{width}|,表示将内容居中,且列宽度为|width|。
%
% |array|环境中的\cs{centering}命令会改变\cs{newline}的定义,因此我们需要用
% \cs{arraybackslash}将其恢复;另外,我们也可能会在列内容中使用\cs{newline},因此在
% \cs{centering}后重新定义了\cs{newline}。
%
%    \begin{macrocode}
\newcolumntype{C}[1]{>{\centering\let\newline\\%
    \arraybackslash\hspace{0pt}}p{#1}}
%    \end{macrocode}
% \end{macro}
%
% \begin{environment}{arabicenum}
% 阿拉伯数字列表环境。该列表最多三层。
%    \begin{macrocode}
\newlist{arabicenum}{enumerate}{3}
\setlist[arabicenum,1]{label=\textnormal%
  {\textnormal{(\arabic*)}}}
\setlist[arabicenum,2]{label=\textnormal%
  {\textnormal{(\arabic{arabicenumi}.\arabic*)}}}
\setlist[arabicenum,3]{label=\textnormal%
  {\textnormal{(\arabic{arabicenumi}.\arabic{arabicenumii}.\arabic*)}}}
%    \end{macrocode}
% \end{environment}
%
% \begin{environment}{romanenum}
% 罗马数字列表环境。该列表最多两层。
%    \begin{macrocode}
\newlist{romanenum}{enumerate}{2}
\setlist[romanenum,1]{label={\textnormal{\roman*.}}}
\setlist[romanenum,2]{label={\textnormal{\alph*\,)}}}
%    \end{macrocode}
% \end{environment}
%
% \begin{environment}{alphaenum}
% 小写字母列表环境。该列表最多两层。
%    \begin{macrocode}
\newlist{alphaenum}{enumerate}{2}
\setlist[alphaenum,1]{label={\textnormal{\alph*\,)}}}
\setlist[alphaenum,2]{label={\textnormal{\alph{alphaenumi}.\arabic*\,)}}}
%    \end{macrocode}
% \end{environment}
%
% \begin{environment}{caseenum}
% 情况分类列表环境。该列表最多两层。
%    \begin{macrocode}
\newlist{caseenum}{enumerate}{2}
\setlist[caseenum,1]{label={\textnormal{\zhbook@cap@case\arabic*.}}}
\setlist[caseenum,2]{label={\textnormal{\zhbook@cap@subcase\arabic{caseenumi}.\arabic*.}}}
\setlist[caseenum]{leftmargin=*}
%    \end{macrocode}
% \end{environment}
%
% \begin{environment}{stepenum}
% 步骤列表环境。该列表最多两层。
%    \begin{macrocode}
\newlist{stepenum}{enumerate}{2}
\setlist[stepenum,1]{label={\textnormal{\zhbook@cap@step\arabic*.}}}
\setlist[stepenum,2]{label={\textnormal{\zhbook@cap@substep\arabic{stepenumi}.\arabic*.}}}
\setlist[stepenum]{leftmargin=*}
%    \end{macrocode}
% \end{environment}
%
% \subsection{设置PDF文档属性}
%
% \begin{macro}{\zhbook@setpdfinfo}
% 此命令设置PDF文档属性,依赖于|hyperref|宏包。
%    \begin{macrocode}
\newcommand*{\zhbook@setpdfinfo}{\hypersetup{%
        pdftitle={\zhbook@value@title},
        pdfauthor={\zhbook@value@author},
        pdfkeywords={\zhbook@value@abstract@keywords},
        pdfcreator={\zhbook@value@author},
        pdfproducer={XeLaTeX with the Zh-Book document class}}
}
%    \end{macrocode}
% \end{macro}
%
% 在文档的\cs{begin{document}}之后立即调用\cs{njut@setpdfinfo}命令设置PDF文档属性。
%    \begin{macrocode}
\AtBeginDocument{\zhbook@setpdfinfo}
%</cls>
%    \end{macrocode}
% \StopEventually{\PrintIndex}
% \Finale
%
% \iffalse
%    \begin{macrocode}
%<*dtx-style>
\ProvidesPackage{dtx-style}
\RequirePackage{amssymb}
\RequirePackage{calc}
\RequirePackage{array,longtable}
\RequirePackage{fancybox,fancyvrb}
\RequirePackage{xcolor}
\RequirePackage{txfonts}
\RequirePackage{xltxtra}
\RequirePackage{subfigure}
\RequirePackage{marvosym}
\RequirePackage{booktabs}
\RequirePackage{paralist}
\RequirePackage{enumitem}
\RequirePackage{titlesec}
\RequirePackage{titling}
\RequirePackage{fancyhdr}
\RequirePackage{geometry}
\RequirePackage{indentfirst}
\RequirePackage[CJKnumber,CJKchecksingle]{xeCJK}
\RequirePackage[hyphens]{url} % must be load before hypdoc package
\RequirePackage{hypdoc} % it will load hyperref package
\RequirePackage[normalem]{ulem}

\hypersetup{%
    unicode=true,
    hyperfootnotes=true,
    hyperindex=true,
    pageanchor=true,
    CJKbookmarks=true,
    bookmarksnumbered=true,
    bookmarksopen=true,
    bookmarksopenlevel=0,
    breaklinks=true,
    colorlinks=false,
    plainpages=false,
    pdfpagelabels,
    pdfborder=0 0 0%
}

\newcommand{\env}[1]{\texttt{#1}}

% 设置超链接为蓝色
\hypersetup{colorlinks=true,urlcolor=blue}

% 定义英文字体名称。
\newcommand*{\zhbook@enfn@main}{Times New Roman}
\newcommand*{\zhbook@enfn@sans}{Arial}
\newcommand*{\zhbook@enfn@mono}{Courier New}

% 选择中文字体
\newcommand*{\zhbook@zhfn@songti}{Adobe Song Std}
\newcommand*{\zhbook@zhfn@heiti}{Adobe Heiti Std}
\newcommand*{\zhbook@zhfn@kaishu}{Adobe Kaiti Std}
\newcommand*{\zhbook@zhfn@fangsong}{Adobe Fangsong Std}

% 定义中文字体
\setCJKfamilyfont{song}{\zhbook@zhfn@songti}
\setCJKfamilyfont{hei}{\zhbook@zhfn@heiti}
\setCJKfamilyfont{kai}{\zhbook@zhfn@kaishu}
\setCJKfamilyfont{fangsong}{\zhbook@zhfn@fangsong}

\setCJKmainfont[BoldFont={\zhbook@zhfn@heiti},
                ItalicFont={\zhbook@zhfn@kaishu}]{\zhbook@zhfn@songti}
\setCJKsansfont{\zhbook@zhfn@heiti}
\setCJKmonofont{\zhbook@zhfn@fangsong}

% 定义文档使用的英文字体
\setmainfont{\zhbook@enfn@main}
\setsansfont{\zhbook@enfn@sans}
\setmonofont{\zhbook@enfn@mono}

% 定义中文字体选择命令
\newcommand*{\songti}{\CJKfamily{song}}
\newcommand*{\heiti}{\CJKfamily{hei}}
\newcommand*{\kaishu}{\CJKfamily{kai}}
\newcommand*{\fangsong}{\CJKfamily{fangsong}}

\renewcommand{\contentsname}{目\hspace{2em}录}
\renewcommand{\abstractname}{摘\hspace{2em}要}
\renewcommand{\indexname}{索\hspace{2em}引}
\renewcommand{\figurename}{图}
\renewcommand{\tablename}{表}
\renewcommand{\refname}{参考文献}

\setlength{\parskip}{4pt plus1pt minus0pt}
\setlength{\topsep}{0pt}
\setlength{\partopsep}{0pt}
\setlength{\parindent}{2em}
\addtolength{\oddsidemargin}{-1cm}
\advance\textwidth 1.5cm
\addtolength{\topmargin}{-1cm}
\addtolength{\headsep}{0.3cm}
\addtolength{\textheight}{2.3cm}

\newcommand{\zhdash}{\kern0.3ex\rule[0.8ex]{2em}{0.1ex}\kern0.3ex}

\renewcommand{\baselinestretch}{1.3}

\DefineVerbatimEnvironment{shell}{Verbatim}%
  {frame=single,framerule=0.1mm,rulecolor=\color{black},%
   framesep=2mm,fontsize=\small,gobble=1}

\DefineVerbatimEnvironment{example}{Verbatim}%
  {frame=single,framerule=0.1mm,rulecolor=\color{black},%
   framesep=2mm,baselinestretch=1.2,fontsize=\small,gobble=1}

\long\def\myentry#1{\vskip5pt\par\noindent\llap{{\color{blue}\fangsong #1}}%
  \marginpar{\strut}\hskip\parindent}

% 使用|titlesec|宏包提供的\titleformat命令设置标题格式:
\titleformat*{\section}{\Large\bfseries}
\titleformat*{\subsection}{\large\bfseries}
\titleformat*{\subsubsection}{\normalsize\bfseries}
\titleformat*{\paragraph}{\normalsize\bfseries}
\titleformat*{\subparagraph}{\normalsize\bfseries}

% 使用|titling|宏包设置标题的字体
\pretitle{\begin{center}\huge\bfseries}
\posttitle{\par\end{center}\vskip 1em}
\preauthor{\begin{center}
             \large \lineskip 0.5em}
\postauthor{\par\end{center}}
\predate{\begin{center}\large}
\postdate{\par\end{center}}

% 修改\cs{tableofcontents}命令用于生成目次页。
\renewcommand{\tableofcontents}{%
    \if@twocolumn
      \@restonecoltrue\onecolumn
    \else
      \@restonecolfalse
    \fi
    \section*{\hfill\contentsname\hfill}%
    \@mkboth{\MakeUppercase\contentsname}{\MakeUppercase\contentsname}%
    \@starttoc{toc}%
    \if@restonecol\twocolumn\fi
}

% 增加一种新的表格列对齐方式 C{width},表示该列内容居中且宽度为width
\newcolumntype{C}[1]{>{\centering\let\newline\\%
    \arraybackslash\hspace{0pt}}p{#1}}


% \dangericon 表示警告的图标
\font\manfnt=manfnt
\newcommand*{\dangericon}{\manfnt\char127}

% note 环境表示需特别注意的内容
\newenvironment{note}
               {\vskip1.5ex\par\noindent\llap{\dangericon\hskip2mm}\textbf{注意:}}
               {\vskip1.5ex}

% syntax 环境表示语法描述
\newenvironment{syntax}
               {\begin{center}}
               {\end{center}}

\newenvironment{suggestion}
               {\par\noindent\textbf{建议:}}{}

% 重新设置默认的列表的格式。
\setlist{%
  topsep=0.3em,             % 列表顶端的垂直空白
  partopsep=0pt,            % 列表环境前面紧接着一个空白行时其顶端的额外垂直空白
  itemsep=0ex plus 0.1ex,   % 列表项之间的额外垂直空白
  parsep=0pt,               % 列表项内的段落之间的垂直空白
  leftmargin=1.5em,           % 环境的左边界和列表之间的水平距离
  rightmargin=0em,          % 环境的右边界和列表之间的水平距离
  labelsep=0.5em,           % 包含标签的盒子与列表项的第一行文本之间的间隔
  labelwidth=2em,           % 包含标签的盒子的正常宽度;若实际宽度更宽,则使用实际宽度。
}

% 设置无序列表的标签符号。
\setlist[itemize,1]{label=$\bullet$}
\setlist[itemize,2]{label=$\blacksquare$}
\setlist[itemize,3]{label=$\Diamondblack$}

% 默认的|fancy|风格会在每一章最后的空白页(由于是双面打印)也加上页眉页脚,但我
% 们通常不希望如此。解决方法是修改{\LaTeX}内部的\cleardoublepage命令的定义如下:
\makeatletter
\def\cleardoublepage{\clearpage\if@twoside \ifodd\c@page\else
  \hbox{}\thispagestyle{empty}\newpage\if@twocolumn\hbox{}\newpage\fi\fi\fi}
\makeatother

% 文章用A4纸标准大小的白纸打印,页眉:2.6cm,页脚:2.4cm,页边距上下:3.5cm,左
% 右:3.2cm。
\geometry{headheight=2.6cm,headsep=3mm,footskip=13mm}
\geometry{top=3.5cm,bottom=3.5cm,left=3.2cm,right=3.2cm}


% \std{code}表示国家标准编号
\newcommand*{\std}[1]{{\normalfont #1}}

% 增加环境命令: \env{name} 表示名为 name 的环境
%% \newcommand{\env}[1]{\texttt{#1}}

% 修改\tableofcontents命令用于生成目次页,将目次页本身也被加入目次中。
\makeatletter
\renewcommand*{\tableofcontents}{%
    \if@twocolumn
      \@restonecoltrue\onecolumn
    \else
      \@restonecolfalse
    \fi
    \section*{\hfill\contentsname\hfill}%
    \@mkboth{\MakeUppercase\contentsname}{\MakeUppercase\contentsname}%
    \addcontentsline{toc}{section}{\contentsname}%
    \@starttoc{toc}%
    \if@restonecol\twocolumn\fi
}
\makeatother

% 设置索引页面的样式
\IndexPrologue{\clearpage\section*{\hfill\indexname\hfill}%
\markboth{\indexname}{\indexname}%
\addcontentsline{toc}{section}{\indexname}%
斜体数字表示对应项的描述所在页面的页码, %
带下划线的数字表示对应项的定义所在的代码行号,%
其他数字表示对应项所被引用的代码行号。%
}

% 设置索引页面的栏数
\setcounter{IndexColumns}{2}


\newcommand{\dashnumber}[2]%
  {{#1}\kern.07em\rule[.5ex]{.4em}{.1ex}\kern.07em{#2}}

%</dtx-style>
%    \end{macrocode}
% \fi
\endinput
