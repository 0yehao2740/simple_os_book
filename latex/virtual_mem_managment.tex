\section{【原理】虚拟内存管理}\label{ux539fux7406ux865aux62dfux5185ux5b58ux7ba1ux7406}

什么是虚拟内存?简单地说,是指程序员或CPU
``需要''和直接``看到''的内存,这其实暗示了两点:1、虚拟内存单元不一定有实际的物理内存单元对应,即实际的物理内存单元可能不存在;2、如果虚拟内存单元对应有实际的物理内存单元,那二者的地址一般不是相等的。通过操作系统的某种内存管理和映射技术可建立虚拟内存与实际的物理内存的对应关系,使得程序员或CPU访问的虚拟内存地址会转换为另外一个物理内存地址。

那么这个``虚拟''的作用或意义在哪里体现呢?在操作系统中,虚拟内存其实包含多个虚拟层次,在不同的层次体现了不同的作用。首先,在有了分段或分页机制后,程序员或CPU直接``看到''的地址已经不是实际的物理地址了,这已经有一层虚拟化,我们可简称为内存地址虚拟化。有了内存地址虚拟化,我们就可以通过设置段界限或页表项来设定软件运行时的访问空间,确保软件运行不越界,完成内存访问保护的功能。

通过内存地址虚拟化,可以使得软件在没有访问某虚拟内存地址时不分配具体的物理内存,而只有在实际访问某虚拟内存地址时,操作系统再动态地分配物理内存,建立虚拟内存到物理内存的页映射关系,这种技术属于lazy
load技术,简称按需分页(demand
paging)。把不经常访问的数据所占的内存空间临时写到硬盘上,这样可以腾出更多的空闲内存空间给经常访问的数据;当CPU访问到不经常访问的数据时,再把这些数据从硬盘读入到内存中,这种技术称为页换入换出(page
swap
in/out)。两个虚拟页的数据内容相同时,可只分配一个物理页框,这样如果对两个虚拟页的访问方式是只读方式,这这两个虚拟页可共享页框,节省内存空间;如果CPU对其中之一的虚拟页进行写操作,则这两个虚拟页的数据内容会不同,需要分配一个新的物理页框,并将物理页框标记为可写,这样两个虚拟页面将映射到不同的物理页帧,确保整个内存空间的正确访问。这种技术称为写时复制(Copy
On
Write,简称COW)。这三种内存管理技术给了程序员更大的内存``空间'',我们称为内存空间虚拟化。

ucore在实现上述三种技术时,需要解决的一个关键问题是,何时进行请求调页/页换入换出/写时复制处理?其实,在程序的执行过程中由于某种原因(页框不存在/写只读页等)而使
CPU
无法最终访问到相应的物理内存单元,即无法完成从虚拟地址到物理地址映射时,CPU
会产生一次缺页异常,从而需要进行相应的缺页异常服务例程。这个缺页异常处理的时机就是求调页/页换入换出/写时复制处理的执行时机,当相关处理完成后,缺页异常服务例程会返回到产生异常的指令处重新执行,使得软件可以继续正常运行下去。
