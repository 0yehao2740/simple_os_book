\section{【实现】vma\_struct数据结构和相关操作}\label{ux5b9eux73b0vmaux5fstructux6570ux636eux7ed3ux6784ux548cux76f8ux5173ux64cdux4f5c}

在讲述缺页异常处理前,需要建立好虚拟内存空间描述。在proj7之前有关内存的数据结构和相关操作都是直接针对实际存在的资源--物理内存空间的管理,没有从一般应用程序对内存的``需求''考虑,即需要有相关的数据结构和操作来体现一般应用程序对虚拟内存的``需求''。一般应用程序的对虚拟内存的``需求''与物理内存空间的``供给''没有直接的对应关系,ucore是通过缺页异常处理来间接完成这二者之间的衔接。

在ucore中描述应用程序对虚拟内存``需求''的数据结构是vma\_struct,以及针对vma\_struct的函数操作。这里把一个vma\_struct结构的变量简称为vma变量。vma\_struct的定义如下:

\begin{lstlisting}
struct vma_struct {
    struct mm_struct *vm_mm;
    uintptr_t vm_start;
    uintptr_t vm_end;
    uint32_t vm_flags;
    list_entry_t list_link;
};
\end{lstlisting}

vm\_start和vm\_end描述了一个连续地址的虚拟内存空间的起始位置和结束位置,这两个值都应该是
PGSIZE 对齐的,而且描述的是一个合理的地址空间范围(即严格确保 vm\_start
\textless{} vm\_end
的关系);list\_link是一个双向链表,按照从小到大的顺序把一系列用vma\_struct表示的虚拟内存空间链接起来,并且还要求这些链起来的
vma\_struct
应该是不相交的,即vma之间的地址空间无交集;vm\_flags表示了这个虚拟内存空间的属性,目前的属性包括:

\begin{lstlisting}
#define VM_READ 0x00000001 //只读
#define VM_WRITE    0x00000002 //可读写
#define VM_EXEC 0x00000004 //可执行
\end{lstlisting}

以后还会引入如 VM\_STACK 的其它属性来支持动态扩展用户栈空间。

vm\_mm是一个指针,指向一个比vma\_struct更高的抽象层次的数据结构mm\_struct,这里把一个mm\_struct结构的变量简称为mm变量。这个数据结构表示了包含所有虚拟内存空间的共同属性,具体定义如下

\begin{lstlisting}
struct mm_struct {
    list_entry_t mmap_list;
    struct vma_struct *mmap_cache;
    pde_t *pgdir;
    int map_count;
};
\end{lstlisting}

mmap\_list是双向链表头,链接了所有属于同一页目录表的虚拟内存空间,mmap\_cache是指向当前正在使用的虚拟内存空间,由于操作系统执行的``局部性''原理,当前正在用到的虚拟内存空间在接下来的操作中可能还会用到,这时就不需要查链表,而是直接使用此指针就可找到下一次要用到的虚拟内存空间。由于
mmap\_cache 的引入,使得 mm\_struct 数据结构的查询加速 30\% 以上。pgdir
所指向的就是 mm\_struct
数据结构所维护的页表。通过访问pgdir可以查找某虚拟地址对应的页表项是否存在以及页表项的属性等。map\_count记录
mmap\_list 里面链接的 vma\_struct 的个数。

涉及mm\_struct的操作函数比较简单,只有mm\_create和mm\_destroy两个函数,从字面意思就可以看出是是完成mm\_struct结构的变量创建和删除。在mm\_create中用kmalloc分配了一块空间,所以在mm\_destroy中也要对应进行释放。在ucore运行过程中,会产生描述虚拟内存空间的vma\_struct结构,所以在mm\_destroy中也要进对这些mmap\_list中的vma进行释放。涉及vma\_struct的操作函数也比较简单,主要包括三个:

\begin{itemize}
\tightlist
\item
  vma\_create--创建vma
\item
  insert\_vma\_struct--插入一个vma
\item
  find\_vma--查询vma。
\end{itemize}

vma\_create函数根据输入参数vm\_start、vm\_end、vm\_flags来创建并初始化描述一个虚拟内存空间的vma\_struct结构变量。insert\_vma\_struct函数完成把一个vma变量按照其空间位置{[}vma-\textgreater{}vm\_start,vma-\textgreater{}vm\_end{]}从小到大的顺序插入到所属的mm变量中的mmap\_list双向链表中。find\_vma根据输入参数addr和mm变量,查找在mm变量中的mmap\_list双向链表中某个vma包含此addr,即vma-\textgreater{}vm\_start\textless{}=
addr end。这三个函数与后续讲到的缺页异常处理有紧密联系。
