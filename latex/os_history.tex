\subsection{操作系统历史}

在大众的眼中,操作系统就是他们的手机/终端上的软件系统,包括各种应用程序集合,但在历史上,操作系统也是从无到有地逐步发展起来的。操作系统主要完成对硬件控制和对应用程序的服务所必需的功能,操作系统的历史与计算机发展的历史密不可分。操作系统的内涵和功能随着历史的发展也在一直变化,改进中,在今天,没有图形界面和各种文件浏览器已经不能称为一个操作系统了。

\subsubsection{三叶虫时代}

计算机在最开始出现的时候是没有操作系统的。启动,扳开关,装卡片/纸带等比较辛苦的工作都是计算机操作员(Operator)或者用户自己完成。操作员/用户带着记录有程序和数据的卡片(punch card)或打孔纸带去操作机器。装好卡片/纸带后,启动卡片/纸带阅读器,让计算机把程序和数据读入计算机机的内存中后,计算机就开始工作,并把结果也输出到卡片/纸带或显示屏上,最后程序停止。

由于人的操作效率太低,计算机的机时宝贵,所以就引入监控程序(Monitor)辅助完成输入,输出,加载,运行程序等工作,这是现代操作系统的起源。一般情况下,计算机每次只能执行一个任务,CPU大部分时间都在等待人的缓慢操作。

\subsubsection{恐龙时代}

早期的操作系统非常多样化,专用化,生产商生产出针对各自硬件的专用操作系统,大部分用汇编语言编写,这导致操作系统的进化比较缓慢,但进化再持续。在1964年,IBM公司开发了面向System/360系列机器的统一可兼容的操作系统——OS/360。OS/360是一种批处理操作系统。为了能充分地利用计算机系统,应尽量使该系统连续运行,减少空闲时间,所以批处理操作系统把一批作业(古老的术语,可理解为现在的程序)以脱机方式输入到磁带上,并使这批作业能一个接一个地连续处理:1)将磁带上的一个作业装入内存;2)并把运行控制权交给该作业;3)当该作业处理完成后,把控制权交还给操作系统;4)重复1-3的步骤。

批处理操作系统分为单道批处理系统和多道批处理系统。单道批处理操作系统只能管理内存中的一个(道)作业,无法充分利用计算机系统中的所有资源,致使系统整体性能较差。多道批处理操作系统能管理内存中的多个(道)作业,可比较充分地利用计算机系统中的所有资源,提升系统整体性能。二者的共同特点是人机交互性差,这对修改和调试程序很不方便。

\subsubsection{哺乳动物时代}

20世纪60年代末,提高人机交互方式的分时操作系统越来越展露头角。分时是指多个用户和多个程序以很小的时间间隔来共享使用同一台计算机上的CPU和其他硬件/软件资源。1964年由贝尔实验室、麻省理工学院及美国通用电气公司所共同参与研发目标远大的MULTICS(MULTiplexed Information and Computing System)操作系统,MULTICS是一套安装在大型主机上多人多任务的操作系统。 MULTICS以兼容分时系统(CTSS)做基础,建置在美国通用电力公司的大型机GE-645,目标是连接1000部终端机,支持300的用户同时上线。因MULTICS项目的工作进度过于缓慢,1969年AT\&T的 Bell 实验室从MULTICS 研发中撤出。但贝尔实验室的两位软件工程师 Thompson 与 Ritchie借鉴了一些重要的Multics理念,以C语言为基础,发展出UNIX操作操作系统。UNIX操作系统的早期版本是完全免费的,可以轻易获得并随意修改,所以它得到了广泛的接受。后来,它成为开发小型机操作系统的起点。由于早期的广泛应用,它已经成为的分时操作系统的典范。

\subsubsection{古猿人时代}

20世纪70年代,微型处理器的发展使计算机的应用普及至中小企及个人爱好者,推动了个人计算机(Personal Computer)的发展,也进一步推动了面向个人使用的操作系统的出现。其代表是由微软公司中在20世纪80年代为个人计算机开发的DOS/Windows操作系统,其特点是简单易用,特别是基于Windows操作系统的GUI界面,极大地简化了一般用户使用计算机的难度,使得计算机得到了快速的普及。这里需要注意的是,第一个带GUI界面的个人计算机原型起源于伟大却又让人扼腕叹息的施乐帕洛阿图研究中心PARC(Palo Alto Research Center),PARC研发出的带有图标、弹出式菜单和重叠窗口的GUI(Graphical User Interface),可利用鼠标的点击动作来进行操控,这是当今我们所使用的GUI系统的基础。

\subsubsection{智人时代}

智人时代的操作系统的代表是Linux操作系统内核(Linux kernel),它覆盖了工业控制、物联网、移动终端、桌面计算机、服务器、数据中心到超级计算机的各个领域。其中面向移动终端的Android操作系统基于Linux kernel,已成为21世纪个人终端操作系统的代表之一,Linux kernel在巨型机到数据中心服务器操作系统中也占据了统治地位。

1991年8 月,芬兰学生 Linus Torvalds(中文名:林纳斯·托瓦兹)在 comp.os.minix 新闻组贴上了以下这段话: 

"你好,所有使用 minix 的人 -我正在为 386 ( 486 ) AT 做一个免费的操作系统 ( 只是为了爱好 )......"

而他所说的"爱好″就变成我们今天知道的 Linux kernel。 Linus Torvalds采用了GPL版权协议,通过Internet发布了 Linux kernel的源代码。在Internet的日渐盛行以及 Linux 开放自由的GPL版权之下,吸引了无数计算机Hacker和公司投入开发、改善Linux kernel,使得 Linux kernel的功能日见强大并被广泛使用。 

\subsubsection{神人时代}

当前,大数据、人工智能、机器学习、高速网络、AR/VR对操作系统等系统软件带来了新的挑战。如何有效支持和利用这些技术是未来操作系统的方向。\emph{注:面向此时代的操作系统目前还未出现。}


