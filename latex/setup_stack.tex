\section{【实现】设置栈}\label{ux5b9eux73b0ux8bbeux7f6eux6808}

只有设置好的合适大小和地址的栈内存空间(简称栈空间),才能有效地进行函数调用。这里为了减少汇编代码量,我们就通过C代码来完成显示。由于需要调用C语言的函数,所以需要自己建立好栈空间。设置栈的代码如下:

\begin{lstlisting}
movl    $start, %esp
\end{lstlisting}

由于start位置(0x7c00)前的地址空间没有用到,所以可以用来作为bootloader的栈,需要注意栈是向下长的,所以不会破坏start位置后面的代码。在后面的小节还会对栈进行更加深入的讲解。我们可以通过用gdb调试bootloader来进一步观察栈的变化:

\textbf{【实验】用gdb调试bootloader观察栈信息 }

\begin{enumerate}
\def\labelenumi{\arabic{enumi}.}
\item
  开两个窗口;在一个窗口中,在proj1目录下执行命令make;
\item
  在proj1目录下执行 ``qemu -hda bin/ucore.img -S
  -s'',这时会启动一个qemu窗口界面,处于暂停状态,等待gdb链接;
\item
  在另外一个窗口中,在proj1目录下执行命令 gdb obj/bootblock.o;
\item
  在gdb的提示符下执行如下命令,会有一定的输出:

\begin{lstlisting}
    (gdb) target remote :1234   #与qemu建立远程链接
    (gdb) break bootasm.S:68    #在bootasm.S的第68行“movl $start, %esp”设置一个断点
    (gdb) continue              #让qemu继续执行  
\end{lstlisting}

  这时qemu会继续执行,但执行到bootasm.S的第68行时会暂停,等待gdb的控制。这时可以在gdb中继续输入如下命令来分析栈的变化:

\begin{lstlisting}
    (gdb) info registers esp
    esp            0xffd6   0xffd6    #没有执行第68行代码前的esp值
    (gdb) si                          #执行第68行代码
    69        call bootmain
    (gdb) info registers esp
    esp            0x7c00   0x7c00   #当前的esp值,即栈顶
    (gdb) si
    bootmain () at boot/bootmain.c:87    #执行call汇编指令
    87      bootmain(void) {
    (gdb) info registers esp
    esp            0x7bfc   0x7bfc    #当前的esp值0x7bfc, 0x7bfc处存放了bootmain函数的返回地址0x7c4a,这可以通过下面两个命令了解  
    (gdb) x /4x 0x7bfc                  
    0x7bfc: 0x00007c4a      0xc031fcfa      0xc08ed88e      0x64e4d08e
    (gdb) x /4i 0x7c40
       0x7c40 <protcseg+14>:        mov    $0x7c00,%esp
       0x7c45 <protcseg+19>:        call   0x7c6c <bootmain>
       0x7c4a <spin>:       jmp    0x7c4a <spin>
       0x7c4c <gdt>:        add    %al,(%eax)
\end{lstlisting}
\end{enumerate}

\subsection{【提示】}\label{ux63d0ux793a}

在proj1中执行

\begin{lstlisting}
    make debug
\end{lstlisting}

则自动完成上述大部分前期工作,即qemu和gdb的加载,且gdb会自动建立于qemu的联接并设置好断点。具体实现可参看proj1的Makefile中于debug相关的内容和tools/gdbinit中的内容。
