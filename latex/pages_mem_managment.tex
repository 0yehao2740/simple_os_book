\section{【原理】分页内存管理}\label{ux539fux7406ux5206ux9875ux5185ux5b58ux7ba1ux7406}

在分页内存管理中,一方面把实际物理内存(也称主存)划分为许多个固定大小的内存块,称为物理页面,或者是页框(page
frame);另一方面又把CPU(包括程序员)看到的虚拟地址空间也划分为大小相同的块,称为虚拟页面,或者简称为页面、页(page)。页面的大小要求是2的整数次幂,一般在256个字节到4M字节之间。在本书中,页面的大小设定为4KB。在32位的86x86中,虚拟地址空间是4GB,物理地址空间也也是4GB,因此在理论上程序可访问到1M个虚拟页面和1M个物理页面。软件的每一物理页面都可以放置在主存中的任何地方,分页系统(需要CPU等硬件系统提供相应的分页机制硬件支持,详见下一节)提供了程序中使用的虚地址和主存中的物理地址之间的动态映射。这样当程序访问一个虚拟地址时,支持分页机制的相关硬件自动把CPU访问的虚拟地址虚拟地址拆分为页号(可能有多级页号)和页内偏移量,再把页号映射为页帧号,最后加上页内偏移组成一个物理地址,这样最终完成对这个地址的读/写/执行等操作。

假设程序在运行时要去读地址0x100的内容到寄存器1(用REG1表示)中,执行如下的指令:

\begin{lstlisting}
mov 0x100, REG1
\end{lstlisting}

虚拟地址0x100被发送给CPU内部的内存管理单元(MMU),然后MMU通过支持分页机制的相关硬件逻辑就会把这个虚拟地址是位于第0个虚拟页面当中(设页大小为4KB),页内偏移是0x100;而操作系统的分页管理子系统已经设置好第0个虚拟页面对应的是第2个物理页帧,物理页帧的起始地址是0x2000,然后再加上页内的偏移地址0x100,所以最后得到的物理地址就是0x2100。然后MMU就会把这个真正的物理地址发送到计算机系统中的地址总线上,从而可正确访问相应的物理内存单元。

如果操作系统的分页管理子系统没有设置第0个虚拟页面对应的物理页帧,则表示第0个虚拟页面当前没有对应的物理页帧,这会导致CPU产生一个缺页异常,由操作系统的缺页处理服务例程来选择如何处理。如果缺页处理服务例程认为这是一次非法访问,它将报错,终止软件运行;如果它认为是一次合理的访问,则它会采用分配物理页等手段建立正确的页映射,使得能够重新正确执行产生异常的访存指令。
